%%%%%%%%%%%%%%%%%%%%%%%%%%  phdsymp_sample2e.tex %%%%%%%%%%%%%%%%%%%%%%%%%%%%%%
%% changes for phdsymp.cls marked with !PN
%% except all occ. of phdsymp.sty changed phdsymp.cls
%%%%%%%%%%                                                       %%%%%%%%%%%%%
%%%%%%%%%%    More information: see the header of phdsymp.cls   %%%%%%%%%%%%%
%%%%%%%%%%                                                       %%%%%%%%%%%%%
%%%%%%%%%%%%%%%%%%%%%%%%%%%%%%%%%%%%%%%%%%%%%%%%%%%%%%%%%%%%%%%%%%%%%%%%%%%%%%%


%\documentclass[10pt]{phdsymp} %!PN
\documentclass[twocolumn]{phdsymp} %!PN
%\documentclass[12pt,draft]{phdsymp} %!PN
%\documentstyle[twocolumn]{phdsymp}
%\documentstyle[12pt,twoside,draft]{phdsymp}
%\documentstyle[9pt,twocolumn,technote,twoside]{phdsymp}

\usepackage{times}


\def\BibTeX{{\rm B\kern-.05em{\sc i\kern-.025em b}\kern-.08em
    T\kern-.1667em\lower.7ex\hbox{E}\kern-.125emX}}

\newtheorem{theorem}{Theorem}

\begin{document}

\title{Using the Document Class phdsymp.cls} %!PN

\author{Joris Thybaut \thanks{J.~Thybaut is with the Chemical
    Engineering Department, Ghent University (UGent), Gent,
Belgium. E-mail: Joris.Thybaut@UGent.be .}}

\supervisor{Eric Laermans, Luc Dupr\'e}

\maketitle

\begin{abstract}
This article explains how to use the \LaTeX\ style recommended for the
Proceedings of the FTW PhD Symposium. The article is itself an example of the
phdsymp.cls style in action.
\end{abstract}

\begin{keywords}
Style file, \LaTeX, FTW PhD Symposium
\end{keywords}

\section{Introduction}
\PARstart{T}{he} Symposium covers a wide range of topics and reflects the
diversity of research activities at our Faculty, such as:

%% NOTE: "\itemindent -1em \leftmargini 2em" . Sometimes other values
%% are used, e.g. "\itemindent 2em \leftmargini 0em"
\begin{itemize}
\item Applied Physics
\item Architecture
\item Automation
\item \ldots{}
\end{itemize}

Authors who have prepared their articles using \LaTeX\
can get them formatted in the style we would like to recommend for the
Proceedings of the Symposium.  The style file {\tt phdsymp.cls} can be used
together with the bibliography style file {\tt phdsymp.bst}.

We recommend a {\em double column} style as this makes the article easier to
read. The column width is 21 pica (approximately 90 mm). In this sample file
you will find examples for the layout of displayed equations, theorems, tables,
figures, etc.
%


\section{How to Use the File phdsymp.cls}

\subsection{General Information}
This style file has been written so to allow, with very few changes,
the formatting of input that is suitable for the \LaTeX\ {\tt article}
style.
First,  the \verb+phdsymp.cls+ style file has to be
selected with a command of the form
\begin{center}
%\verb+\documentstyle[twocolumn]{phdsymp}+ %!PN
\verb+\documentclass[twocolumn]{phdsymp}+
\end{center}

The default font size is 10 points.  
%The default page style has been
%redefined and is now set by {\tt phdsymp.cls} to ``\verb+headings+''.

The Symposium Proceedings will not include author affiliations below or beside
the name(s) of the author(s); instead, use the command \verb+\thanks{...}+ to
list addresses. Note that the \verb+\thanks{..}+ command in the title no longer
produce marks: the thanks-footnote should therefore be self-contained, with
address and name of the author(s).

The command ``\verb+\PARstart{X}{YYY} ZZZ+'' produces a large letter
\verb+X+ at the beginning of the paragraph. The string \verb+YYY+
will be automatically changed to capital letters.

The bibliography style file {\tt phdsymp.bst} allows \BibTeX\ to include
the references from the chosen bibliography file(s) according to the
format recommended for the Symposium Proceedings.

Footnotes produce a footnote mark as usual.\footnote{The footnote is
indicated by a footnote mark}

%Full papers should include the biography of the authour.
%An example of a formatted biography is given at the end of
%this sample article.
%The environment is called \verb+biography+ and requires the
%name of the person whose bio\-graphy is presented.

In figure \ref{fig-example} we can see an example for the definition of
the title page and of the main commands needed to compile a \LaTeX
file with phdsymp.cls.

\begin{figure}[htb]
\mbox{}\hrulefill
\vspace{-.3em}
%\documentstyle[twocolumn,twoside]{phdsymp}
% !PN these three lines out of verbatim...
%\title{Using the \LaTeX Style File phdsymp.sty}
\begin{verbatim}
\documentclass[twocolumn]{phdsymp} 

\usepackage{times}

\begin{document}

\title{Using the Document Class phdsymp.cls}

\author{Joris Thybaut
   \thanks{J.~Thybaut is...}}

\supervisor{Eric Laermans, Luc Dupr\'e}

\maketitle

\begin{abstract}
This article ...
\end{abstract}

\begin{keywords}
Style file...
\end{keywords}

\section{Introduction}
\PARstart{T}{he} Symposium ...

\bibliographystyle{phdsymp}
\bibliography{bib-file}

\end{document}
\end{verbatim}
\vspace{-.6em}
\mbox{}\hrulefill
\caption{Input used to produce this paper.}
\label{fig-example}
\end{figure}

%The command \verb+\markboth{leftTEXT}{rightTEXT}+ can be used for
%setting the running heads. If the option {\tt twoside} is not
%selected, both even and odd headers will display {\tt leftTEXT}
%together with the page number.  Note that the header of the title page
%always displays {\tt leftTEXT} as it bears the journal name.

\subsection{Additional Changes}
Most changes resulting from the use of phdsymp.cls should be
transparent to the user. For instance,
captions for figures and tables have been modified. Caption of
tables, however, should be defined before the table item.

\begin{figure}[hbt]
\begin{center}
\setlength{\unitlength}{0.0105in}%
\begin{picture}(242,156)(73,660)
\put( 75,660){\framebox(240,150){}}
\put(105,741){\vector( 0, 1){ 66}}
\put(105,675){\vector( 0, 1){ 57}}
\put( 96,759){\vector( 1, 0){204}}
\put(105,789){\line( 1, 0){ 90}}
\put(195,789){\line( 2,-1){ 90}}
\put(105,711){\line( 1, 0){ 60}}
\put(165,711){\line( 5,-3){ 60}}
\put(225,675){\line( 1, 0){ 72}}
\put( 96,714){\vector( 1, 0){204}}
\put( 99,720){\makebox(0,0)[rb]{\raisebox{0pt}[0pt][0pt]{\tenrm $\varphi$}}}
\put(291,747){\makebox(0,0)[lb]{\raisebox{0pt}[0pt][0pt]{\tenrm $\omega$}}}
\put(291,702){\makebox(0,0)[lb]{\raisebox{0pt}[0pt][0pt]{\tenrm $\omega$}}}
\put( 99,795){\makebox(0,0)[rb]{\raisebox{0pt}[0pt][0pt]{\tenrm $M$}}}
\end{picture}
\end{center}
\caption{This is a sample figure. The caption comes after the figure.}
\end{figure}

\begin{table}[htb]
\caption{The Caption Comes Before the Table.}
\begin{center}
{\tt
\begin{tabular}{|c||c|c|c|}\hline
&title page&odd page&even page\\\hline\hline
onesided&leftTEXT&leftTEXT&leftTEXT\\\hline
twosided&leftTEXT&rightTEXT&leftTEXT\\\hline
\end{tabular}
}
\end{center}
\end{table}

\subsubsection{Environments}
The environments for theorem, propositions, lemmas, etc. can be
defined with the usual \LaTeX\ \cite{LaTeX,LaTeXD} command
\verb+\newtheorem{..}{..}+.  The proof environment is already defined.

\begin{theorem}[Theorem name]
Consider the system
\begin{equation}
\begin{array}{rrr}
\dot x&=&A.x+B.u\\[2mm]
y&=& C.x+D.u
\end{array}
\end{equation}
If $A$ is stable, then the pair $\{A,B\}$ is stabilizable. Moreover,
this holds for any $B$.
\end{theorem}
\begin{proof}
The proof is trivial.
\end{proof}

%\subsection{Preparing a Technical Note}
%A technical note can be prepared using the additional option
%\verb+technote+ in the documentstyle command. Note that the default
%point size is still 10 points, but that 9 points should be selected.
%The format for a technical note can thus be selected with a command
%of the form
%\begin{center}
%%\verb+\documentstyle[...side,9pt,technote]{phdsymp}+
%\verb+\documentclass[...side,9pt,technote]{phdsymp}+ %!PN
%\end{center}
%All the definitions and commands are still valid  even after the
%changes caused by the option \verb+technote+.

%\subsection{Submitting a Paper}
%The paper can be prepared for submission by omitting the option {\tt
%twocolumn} and choosing the option {\tt draft} (this will modify the
%baselinestretch variable).
%Thus, the format for submission contains a definition of the form
%\begin{center}
%%\verb+\documentstyle[...side,12pt,draft]{phdsymp}+
%\verb+\documentclass[...side,12pt,draft]{phdsymp}+ %!PN
%\end{center}

\section{Optional Formatting}

When you are happy with the {\em ultimate unequivocal final version},
you may perform following additional changes, although this is certainly not
necessary.

\subsection{``Hard-Coding'' Symbolics}
Change the symbolics so that the file actually contains the reference
numbers
{\em i.e.\ \/}``\verb+... \cite{fred:88} ...+'' should be changed to
``\verb+... [3] ...+''. One author (who used the style file) did a smart
thing
{\em after\/} he had decided upon a final version.  He put his
\verb+\cite{..}+ command and other symbolics on a line on their own
and commented them out (from the formatting) by putting a \verb+%+
sign before each symbolic. Then, on the next line he just inserted the
copy-matching numerical, like this:

\begin{verbatim}
    Well, according the Fred Bloggs
    %\cite{fred:88}
    [24]
    the value of $\alpha$ should be even
    greater than what we think it should be.
\end{verbatim}

Thus, {\em he\/} knows he put in the correct (copy-matching) numerical and
the publishing staff can send him back an author-proof that correctly matches
his submission. This is not so useful here as we do not expect you to send the
\LaTeX\-file, but a PDF-version of your article.

The above also applies to the referencing of table and figures
(and any ``auto-numbering'' feature, standard or synonymous with your
system).

Figure captions can be part of the text (in between paragraphs) like this:

\begin{verbatim}
    And in Fig. 3 we see that the
    value of $\alpha$ increases exponentially.

       Fig. 3\quad This is the caption for
       figure 3 showing some $\alpha$.

    And after the caption we continue on
    with the next paragraph, like this.
\end{verbatim}

In essence, by you actually putting in the {\em correct copy
matching\/} numericals so that no problems arise with incomplete files
being sent to the transactions (the wrong \verb+*.bib+, \verb+*.bbl+ files,
the wrong versions of figures etc).  Also, and more importantly, the
numbers that are on your hard-copy (and in the reviewer's hands) will
be the same ones that you receive in your author proof. Again, this is not so
useful here as we do not expect you to send the \LaTeX\-file, but a PDF-version
of your article.

\subsection{Including the Bibliography into the \LaTeX\ Source File}
You can reduce the number of files you have to send to the
publishers in the following way. Run \BibTeX\ on the
\verb+*.aux+ file. This creates a \verb+*.bbl+ file: include this
into your \LaTeX\ source file at the place where you defined the
\verb+\bibliography{..}+ command  and comment this command out.
Remove the \verb+*.bbl+ file.  Then, your \LaTeX\ file will include
all the necessary information about your bibliography and no \verb+*.bbl+
or \verb+*.bib+ file will be needed.

This may seem like an awful lot of work... but not really..
This will allow to process your paper quickly and efficiently, and assure
you that what you send in {\em will\/} actually be sent back to you
without  mistakes (cites, refs etc.).

However, this would only have been useful if you had been requested to send the
\LaTeX\-file itself instead of a PDF-version of the article.

\section{Conclusions}
This sample article has presented the style file phdsymp.cls
This file can be especially useful in preparing articles for
submission to the FTW PhD Symposium.


\section*{Acknowledgments}
The authors would like to acknowledge the suggestions of many people.


\nocite{*}
\bibliographystyle{phdsymp}
%%%%%\bibliography{bib-file}  % commented if *.bbl file included, as
%%%%%see below


%%%%%%%%%%%%%%%%% BIBLIOGRAPHY IN THE LaTeX file !!!!! %%%%%%%%%%%%%%%%%%%%%%%%
%% This is nothing else than the phdsymp_sample2e.bbl file that you would%%
%% obtain with BibTeX: you do not need to send around the *.bbl file        
%%
%%---------------------------------------------------------------------------%%
%
\begin{thebibliography}{1}
\bibitem{LaTeX}
Leslie Lamport,
\newblock {\em A Document Preparation System: \LaTeX, User's Guide and
  Reference Manual},
\newblock Addison Wesley Publishing Company, 1986.
\bibitem{LaTeXD}
Helmut Kopka,
\newblock {\em \LaTeX, eine Einf\"uhrung},
\newblock Addison-Wesley, 1989.
\bibitem{TeX}
D.K. Knuth,
\newblock {\em The {\rm T\kern-.1667em\lower.7ex\hbox{E}\kern-.125emX}book},
\newblock Addison-Wesley, 1989.
\bibitem{METAFONT}
D.E. Knuth,
\newblock {\em The {\rm METAFONT}book},
\newblock Addison Wesley Publishing Company, 1986.
\end{thebibliography}
%
%%---------------------------------------------------------------------------%%

\end{document}

%%%%%%%%%%%%%%%%%%%%%  End of phdsymp_sample2e.tex  %%%%%%%%%%%%%%%%%%%%%%%%%%%
