\documentclass[10pt,a4paper]{article}

%% PACKAGES WITH OPTIONS
\usepackage[english]{babel}
\usepackage[T1]{fontenc}
\usepackage{lmodern}
\usepackage[utf8]{inputenc}
\usepackage[version=4]{mhchem} % Chemistry and isotopes with \ce{} (other option: 'chemformula' \ch{})
\usepackage[numbers]{natbib} % numbers: needed argument, otherwise error
\usepackage[section]{placeins} % Place floats in their own section
\usepackage[nottoc,notlot,notlof]{tocbibind} % Include bibliography in table of contents

%% PACKAGES WITHOUT OPTIONS
\usepackage{adjustbox} % Use \begin{adjustbox}{center} environment for too wide floats
\usepackage{amsmath, amsfonts, amssymb}
\usepackage{ctable} % Usage: \ctable[options]{coldefs like c|c|r|l}{\tnote commands for footnotes}{normal table content}, with \tnote[symbol]{text} footnote definitions, and \tmark[symbol] used to reference the footnote in the normal table content
\usepackage{enumitem} % Lists
\usepackage{flafter} % Place floats after their first appearance in text
\usepackage{float} % [H] modifier for floats etc.
\usepackage{graphicx}
\usepackage{hyperref, cleveref} % Creates links (boxes in various colors around words)
\usepackage{indentfirst} % Always indent the first line of a section
\usepackage{listings} % Typesetting source code
\usepackage{physics} % Vector calculus etc.
\usepackage{siunitx} % Fancy display of SI units
\usepackage{subcaption} % Enables subfigures
\usepackage{titling}
\usepackage{xcolor} % Anything with color e.g. \fcolorbox{black}{red}{}
\usepackage{xurl} % Loads package 'url' and breaks urls nicely

%% PACKAGES SETUP COMMANDS
\hypersetup{colorlinks=true,urlcolor=blue,citecolor=gray}
\captionsetup{width=.8\linewidth}

%% COMMANDS
% Bold and italic vector symbols (preferably use \vb{} instead)
\renewcommand{\vec}[1]{\boldsymbol{#1}}
% Monospaced inline code (for multiline code, use package 'listings')
\newcommand{\code}[1]{\texttt{#1}}
% Equals sign, with number above referencing some equation
\newcommand{\numeq}[1]{\stackrel{\scriptscriptstyle(\mkern-1.5mu#1\mkern-1.5mu)}{=}}
% If-and-only-if sign, with number above referencing some equation
\newcommand{\numiff}[1]{\stackrel{\scriptscriptstyle(\mkern-1.5mu#1\mkern-1.5mu)}{\Leftrightarrow}}
% Numbers a single line in a no-numbering multiline equation* or align*
\newcommand{\numberthis}{\addtocounter{equation}{1}\tag{\theequation}}
% MuMax3
\newcommand{\mumax}{$\mathsf{mumax}^3$}

%% TITLE VARIABLES
\author{Jonathan Maes}
\title{Biaxial nanomagnets as building block for balanced half-adders}


\begin{document}

\begin{titlingpage}
\maketitle
\end{titlingpage}

\newpage
\pagenumbering{roman}

{\hypersetup{linkcolor=black}
\tableofcontents
}
\newpage
\pagenumbering{arabic}

This thesis makes use of the \mumax{} micromagnetic simulation tool~\cite{MuMax3}.
Gilbert is \cite{Gilbert1956} and Lifdau is \cite{LANDAU1992}.

Interesting sources regarding the random thermal switching could be \cite{ThermFluc_SingleDomain, RandomSwitch_MonteCarlo, Nonmonotonic_reversal}, and perhaps \cite{MagDynamics_JumpNoise}.
Exchange bias is explained in \cite{ExchangeBias, ExchangeBias_nanostructures, ExchangeBias_Mechanisms}.
Quantum Cellular Automata (QCA) are discussed in \cite{QCA_Algorithms, QCA_GameOfLife}. Magnetic QCA (MQCA) are discussed in \cite{MQCA_MajorityGate, MQCA_RoomTemp, MQCA_ImageRecognition}.

\section{Introduction}
%\noindent \textit{Note: This section exists to explain nanomagnetic logic to the unfamiliar reader. \textbf{Biaxial island} begins the investigation of the dynamics of a single biaxial island, and is a suitable starting point for those already familiar with nanomagnetic logic.} \bigskip

\noindent An atom may display what is called a \textbf{magnetic dipole moment}, on one hand due to the angular momentum of electrons around the nucleus, and on the other hand due to the intrinsic spin of the elementary particles which make up the atom. The magnetic moment can be defined using the torque the atom experiences in a magnetic field.~\cite{IntroMagneticMaterials} In a material, many atoms are together, and the macroscopic sum of all their magnetic moments is called the magnetization of the material. In a crystalline material, the periodicity of the lattice can give rise to specific orderings of the magnetic moments, like for example ferromagnetism where all individual magnetic moments of neighboring atoms try to align themselves in the same direction. \par 
Such a ferromagnetic material, like iron, nickel or cobalt, consists of many \textbf{magnetic domains}. Each domain, on the order of several tens of micrometers, has a nearly uniform magnetization pointing in a random direction. Because the direction of each domain is random, a ferromagnetic material does not display a macroscopic net magnetization. However, when one fabricates a very small island of several hundred nanometers made of such a ferromagnetic material, it will not be made up of several domains, but rather only one, and will thus have a nearly uniform magnetization. \par
If this island has one or another form of anisotropy, be it due to the crystal structure or the shape of the island, the magnetization will have minimal energy if it lies along a certain axis, called the `easy axis'. \textbf{Shape anisotropy} originates from the observation that the magnetization of a domain preferably orients itself along the long axis of a microscopic structure. An example that is often used is a 2D ellipse, for which the easy axis is equal to the long axis of the ellipse. In general, if a structure has one easy axis (and thus two stable directions), we speak of uniaxial anisotropy. In the case of two easy axes (and thus four stable directions), as will be the main topic of this thesis, this is called biaxial anisotropy. Biaxial shape anisotropy can be achieved with a geometry that is invariant under rotation over \SI{90}{\degree}. \par
In the uniaxial case, there are two stable directions, which can be related to a `0' or `1' bit. Hence, these nanomagnetic islands can be used to create a \textbf{digital logic circuit}. There have been examples in literature where majority logic gates and NAND gates have been created using nanomagnets.~\cite{GYP-18} Biaxial nanomagnets can be used to encode two bits in one island, because those have 4 stable magnetization directions. This should allow for smaller logic gates due to the increased integration density as compared to uniaxial islands. The main goal of this thesis is to realize a half adder using biaxial nanomagnets. A half adder takes two bits as input and yields two bits as output, so it almost feels natural to design it using biaxial nanomagnets instead of uniaxial ones.
% TODO: talk about making logic devices with this
In the following sections, the nature of magnetism and magnetic domains, the concept of quantum cellular automata, signal propagation and techniques for imaging the magnetization will be explained in more detail.

\subsection{Domains}
In 1907, Weiss proposed the idea that a ferromagnetic material contains several uniformly magnetized domains, but that the magnetization of each domain individually is random, thus resulting in a macroscopic zero net magnetization of a large amount of material.~\cite{MuMax3_advances} These domains should not be confused with grains, which are only related to the crystallography. The uniform alignment inside a single domain is due to the quantum mechanical Heisenberg-Dirac exchange interaction.~\cite{MuMax3_advances, heisenberg1928theorie} The magnetostatic energy makes it unfavorable for neighboring moments to align themselves parallel to one another, which competes with the exchange energy which tends to align neighboring moments in the same direction. Since the exchange energy only works on small length scales, domains are formed under the influence of the exchange energy, while different domains tend to cancel out their fields due to the magnetostatic energy. The typical size of these domains is on the order of several tens of micrometers. When the size of the material becomes on the order of micrometers, the domains form very symmetric patterns in order to minimize their energy, and on even smaller length scales only one domain remains.~\cite{NML_Carlton} So, if one makes a droplet of ferromagnetic material smaller than this typical size, the droplet will have a nearly uniform magnetization, with a magnitude equal to the saturation magnetization of the material.~\cite{NML_Carlton} Such a small droplet can for example be produced using electron beam lithography.~\cite{MQCA_RoomTemp, NML_Carlton} By introducing anisotropy in the droplet, which can be either magnetocrystalline or shape anisotropy, one or more axes can be made energetically favorable for the magnetization to align itself along. Magnetocrystalline anisotropy favors the crystal axes. Shape anisotropy can be realized by giving the droplet an elliptical shape instead of a perfect circle. In the uniaxial case, there is one stable magnetization axis, and the two directions `up' and `down' along this axis can be related to bits `0' and `1'.~\cite{MQCA_RoomTemp} This is not to be confused with unidirectional anisotropy, which only has one stable magnetization direction, and can be achieved using the exchange bias effect.~\cite{ExchangeBias_Mechanisms,ExchangeBias_nanostructures,ExchangeBias} A lot of research has been conducted to use uniaxial anisotropy for computation, and logic gates and wires have been proposed, which use classical magnetostatic interactions to propagate the information.~\cite{GYP-18,MQCA_MajorityGate,SwitchingForced_EnergyEfficient} The use of two favorable axes, i.e. biaxial anisotropy, allows for a higher logic density, as the four stable directions `up', `down', `right' and `left' can be related to `00', `01', `10' and `11'.~\cite{MQCA_ImageRecognition} This can occur for cubic crystals, and can also be realized by giving the shape of the droplet more symmetry, for example by making the shape the union of two ellipses.


\subsection{Quantum Cellular Automata}
Traditional digital logic technologies like CMOS use field-effect transistors to control the flow of electrons. This kind of architecture fundamentally only allows information to flow in one direction. Other architectures allowing communication between the output and input in both ways could allow certain types of computations to be executed faster. One approach to realizing such an architecture are the Quantum Cellular Automata (QCA), which use quantum effects, in a broad sense, to make logic gates. A cellular automaton is a mathematical concept, proposed by Von Neumann~\cite{Sideinfo_SelfRepAutomata}, in which the universe is divided into a regular grid of cells, where each cell is influenced by its direct neighbors. The most famous example of such a mathematical cellular automaton is Conway's Game of life, which has been shown to be universal for computation, as any Turing machine can be encoded in it.~\cite{QCA_GameOfLife} Apart from the universality in the Turing sense, cellular automata provide an additional benefit because they are able to process algorithms in a distributed manner due to the spatial parallelism inherent to them.~\cite{QCA_GameOfLife} From a practical point of view, QCA can be orders of magnitude smaller and more energy efficient than traditional CMOS technology.~\cite{MQCA_RoomTemp} \par
There are several possibilities to realize QCA, which can generally be classified as either electronic or magnetic QCA. Electronic QCA (EQCA) make use of the forces between electrons, while magnetic QCA (MQCA) leverage the magnetic moments of atoms. EQCA are called ``quantum'' because they use quantum mechanical tunneling of charge between quantum dots, MQCA are quantum because of the exchange interaction between individual atomic magnetic moments.~\cite{MQCA_RoomTemp} Apart from this quantum nature of the building blocks, QCA are essentially classical devices switching through either coulombic or magnetic interactions.~\cite{QCA_Algorithms} What all these QCA have in common, is that every fundamental building block (electrons, magnetic moments...) of these automata influences every other building block in the automaton, thus allowing the aforementioned two-way flow of information between what would traditionally be referred to as input and output. The MQCA are the main subject of this paper, but it is interesting to take a short look at EQCA as well, for both share some ideas and problems. \par
% Small segue to quantum dot cellular automata
One specific implementation of EQCA makes use of square cells, each with four quantum dots at the vertices of a square. These quantum dots can accomodate an electron, and each cell is made such that there are always two excess electrons present. It is then energetically favorable for these electrons to occupy two diagonal quantum dots.~\cite{QCA_DigitalLogicGate} The two diagonal configurations are then the `0' and `1' states. An extensively studied logic gate using this architecture is the three-input majority logic gate. It consists of five cells arranged in a plus-shape, with three inputs and one output. Just like the familiar NAND gate, the majority gate is universal for computation.~\cite{NML_Carlton} By fixing one of the inputs to 0 or 1, an AND or an OR operation can be realized, respectively. A drawback for these EQCA is that they require low electron temperatures in order to work reliably, as otherwise thermal smearing of the charge states of the dots becomes an issue.~\cite{QCA_DigitalLogicGate} \par
QCA compute by relaxing to a configuration of minimal energy.~\cite{QCA_Algorithms} It is important that all these relaxed configurations are equal in energy level. This is what is meant by `balanced' QCA. A necessary condition for an automaton to be balanced is that the number of distinguished configurations with minimum energy should be equal to the number of input combinations the automaton handles.~\cite{QCA_Algorithms}


\subsection{Nanomagnetic islands}
As was mentioned before, the magnetization of a nanomagnetic island likes to align itself along the longest axis of the shape. This can be understood as follows. Consider a theoretically ideal island so small that it only consists of two atoms, and thus only two magnetic moments. Three different situations are then shown in~\cref{fig:Intro_IslandEllipticPreferredDirection}.
The first configuration has both moments vertical and in the same direction, which makes their magnetic fields nicely lined up, hence this is a low energy state. The second configuration has the magnetic moments in opposite directions, such that their fields oppose each other, which is not a low energy configuration. The third configuration has both moments horizontal and oppositely directed, which once again has the magnetic fields nicely lined up in a low energy state. In the fourth configuration they are pointed horizontal and in the same direction, so the fields once again oppose each other and we do not have a low energy state. Thus, only two of these four configurations are energetically favorable. Now note that we are only dealing with ferromagnetic materials, which have the property that neighboring moments like to align themselves in the same direction, due to the exchange energy (\cref{par:Energy_Exchange}). This only leaves the first situation as a stable one, hence explaining the \textbf{tendency for ferromagnetic materials to be magnetized along the long axis of a microscopic geometric structure}. In larger geometric structures however, the third configuration is a perfectly acceptable way for two separate domains to align themselves, but in this thesis we are only dealing with small single-domain islands. \par
\begin{figure}[t]
    \centering
    \includegraphics[width=0.9\columnwidth]{Figures/Introduction/NML_Carlton - Figure 1.9 adapted.png}
    \caption{Theoretical situation where an island is made up of two magnetic moments, for four different configurations, explaining the tendency for ferromagnetic materials to be magnetized along the long axis of a microscopic geometric structure. Figure adapted from fig. 1.9 in \cite{NML_Carlton}.}
    \label{fig:Intro_IslandEllipticPreferredDirection}
\end{figure}
In case of higher symmetry, for example with biaxial shape anisotropy, it is no longer clear which symmetry axes will be the easy axes, and which the hard ones, as will be examined in more detail in \cref{par:Biaxial_island}.  \par
% TODO: Make this discussion of exchange bias clearer, because it is all over the place now
When constructing a logic gate, it can be useful to include islands whose magnetization direction is permanently fixed, as opposed to the free-switching islands discussed before. One way to realize such a fixed island is through a phenomenon called `exchange bias'.~\cite{ExchangeBias_Mechanisms,ExchangeBias_nanostructures,ExchangeBias,syllabus_PoAEaPD} This phenomenon arises at the interface between an anti-ferromagnetic (AFM) and a ferromagnetic (FM) material due to their exchange anisotropy as illustrated in \cref{fig:Intro_ExchangeBias}.
% TODO: move the mention of 'exchange anisotropy' somewhere else
\begin{figure}[t]
    \centering
    \includegraphics[width=0.9\columnwidth]{Figures/Introduction/Syallabus_PoAEaPD - Figure 2.7.png}
    \caption{The exchange bias effect in ferromagnet/antiferromagnet bilayers with exchange field on the ferromagnetic layer, $H_{ex}$, originating from the interface. Figure taken from \cite{syllabus_PoAEaPD}.}
    \label{fig:Intro_ExchangeBias}
\end{figure}
An AFM material consists of magnetic moments which alternate direction in each atomic plane, while in a FM material all these moments point in the same direction. In the situation as illustrated in the figure, the top layer of the AFM material thus also wants the bottom layer of the FM to align anti-parallel, while the FM layer on the contrary prefers a parallel alignment. One of these two competing effects will be stronger than the other, which often results in an anti-parallel alignment of the magnetic moments at the FM/AFM interface. The AFM material is not affected by external fields since it consists of alternating magnetic moments which leads to a zero net magnetization. The AFM layer will thus remain fixed at temperatures below its N\'{e}el temperature $T_N$, which in turn fixes the FM layer. At strong enough fields, however, the FM layer will follow the external field anyway, though this will occur at significantly higher fields than for the case where there would be no AFM material. Thus, by manufacturing a FM island on top of an AFM material, the island can be made to have a single preferential direction. \par
One might wonder how to define the specific magnetization direction of such an anti-ferromagnetically coupled island in the first place. By choosing a material where the N\'{e}el temperature $T_N$ of the AFM layer is lower than the Curie temperature $T_C$ of the FM layer, one can operate at a temperature $T$ for which $T_N < T < T_C$. In this regime, the FM layer will follow an applied magnetic field, while the AFM layer is paramagnetic due to the high temperature and hence does not influence the hysteresis of the FM layer. When the temperature is then lowered below $T_N$, the AFM material will settle in such a way that its top layer is correctly aligned as to fix the FM layer.~\cite{ExchangeBias_Mechanisms} \par
In general, it will be required to fix different islands in different directions. This can not be achieved by heating the whole substrate and applying an external field, so one has to use a local heating process, for example a focused laser beam. By separately heating each island above $T_N$ and allowing it to cool below $T_N$ before heating the next island, one can still make use of an external field, synchronized with the laser, to control the orientation of each fixed island separately. \par
The switching between stable states of a free FM island without exchange bias can be interpreted as a simple hysteresis of the magnetization $\vb{M}$ as function of the external magnetic field $\vb{H}$. The exchange bias effect shifts this hysteresis loop along the $\vb{H}$-axis, as shown in \cref{fig:Intro_ExchangeBias}. % TODO: talk about how hysteresis of a non-exchange-bias-coupled island before this somewhere

\subsection{Signal Propagation}
Nanomagnetic logic gates need to be able to communicate with each other in order to form a larger and more useful circuit. For this, nanomagnets are placed next to each other, such that they form a sort of chain. Signal propagation through chains of nanomagnets does however come with some large complications. One of the advantages of nanomagnetic logic is that different islands influence each other, but this bidirectional interaction causes problems when trying to propagate signals. Let us start with examining the uniaxial case, as this has been studied most intensively, and then extend our knowledge to biaxial signal propagation.
\subsubsection{Uniaxial}
In the uniaxial case, a wire can be formed by placing elliptic islands next to each other, either along their hard axes (\cref{fig:Intro_IslandEllipticChainGeometries}, left) or along their easy axes (\cref{fig:Intro_IslandEllipticChainGeometries}, right). In the first case, neighboring islands will try to align in opposite directions, in the second case they want to align in the same direction. Let us consider the first case now, as the second case is very similar. % TODO: is the other case extremely similar? I guess so, but I want to see evidence of this because in NML they use additional stabilizers for that configuration, so best to talk about that later.
\begin{figure}
    \centering
    \includegraphics[width=0.5\columnwidth]{Figures/Introduction/Chains_geometries.pdf}
    \caption{Two possible chain geometries for uniaxial islands. Left: chain along the islands' hard axis. Right: chain along the islands' easy axis.}
    \label{fig:Intro_IslandEllipticChainGeometries}
\end{figure}
Suppose the wire has been initialized with a `1'. Then all following moments will be in alternating directions, as shown in the top of \cref{fig:Intro_SolitonRandomWalk}. Now the input bit is changed to a `0' and we wish to propagate this signal through the wire, as shown in the middle of \cref{fig:Intro_SolitonRandomWalk}. The second magnetic moment now wants to align itself up due to its left neighbor (the input bit), but also wants to align itself down due to its right neighbor. As such, no net force acts on the second bit and it will randomly switch. Such a `defect' in the wire is called a magnetic soliton.~\cite{MQCA_RoomTemp} As soon as this second bit changes, the third bit is now in a similar situation, as shown at the bottom of \cref{fig:Intro_SolitonRandomWalk}, and the soliton has moved one place to the right. Now both the second and third bit have an equal chance of switching, and thus the signal has an equal chance of propagating forward to the output (third bit switches) or backward to the input (second bit switches). Thus, the signal (or, equivalently, the soliton) will perform a random walk along the wire. If the input bit is forced to remain `0', the signal will thus reach the output after a time proportional to the square of the number of nanomagnets the chain consists of.~\cite{Wolfram_RandomWalk} It is clear that this is not ideal, and several techniques have been proposed to force the signal to propagate along the wire in one direction. 
\begin{figure}
    \centering
    \includegraphics[width=0.5\columnwidth]{Figures/Introduction/Soliton_random_walk_2steps.pdf}
    \caption{Blue arrows are the input bits. Red arrows have no preferred direction due to competing interactions of their neighbors. A dotted line indicates the location of a magnetic soliton. Top: initial ordering of the wire for a `1' (up) input. Middle: input bit changes to `0' (down), causing no net force to act on the second island. Bottom: second island randomly switches, causing no net force to act on the third island anymore either.}
    \label{fig:Intro_SolitonRandomWalk}
\end{figure}

One technique makes use of an external magnetic field to initialize each island along its hard axis whenever a new bit needs to be propagated. This can be seen as a form of clocking. The first island is initialized in either the `0' or `1' state and functions as an input bit. Keeping this first bit intact under an external field can for example be achieved with the exchange bias effect.~\cite{ExchangeBias_Mechanisms,ExchangeBias_Mechanisms,ExchangeBias} When the external field is removed, each subsequent island chooses a direction to fall onto its easy axis. If this happens adiabatically, each island will fall in the correct direction and no solitons appear.~\cite{NML_Carlton} \par
However, thermal fluctuations can cause some islands to fall in the wrong direction, which can be problematic. It can therefore be instructive to release the islands one at a time to ensure the correct behavior by propagating the signal along the wire one nanomagnet at a time. Furthermore, the use of an external magnetic field to control the wire is not very `clean', because a large external field will affect all the wires in the circuit. One possible solution is to manufacture the nanomagnetic island with a piezoelectric layer underneath, which can induce strain in the island, which exerts a force on the magnetization as will be explained in \cref{par:Energy_MagnetoElastic}. Each nanomagnet can then be controlled electronically. Another possibility is to place the nanomagnetic island between two special electrodes which produce a spin-polarized current. These electrons can then exert a torque on the magnetization, known as the spin transfer torque.~\cite{SwitchingForced_EnergyEfficient,syllabus_PoAEaPD}

\subsubsection{Biaxial}
% TODO: Section on biaxial situation (seems like biaxial is less easy to find information about than uniaxial so need to find some more papers for that)
One way to extract the information from a biaxial island and transmit it along a wire, is to have two wires; one for each component of the orientation.
It is of utmost importance to understand the influence of the wire on the gates, since both influence each other.

Other option is to use elongated biaxial nanomagnets such that the anisotropy caused by the elongated shape in one direction balances the tendency to align in the direction of the wire.

\subsection{Imaging the magnetization}
If one wishes to not only perform theoretical, but also practical studies on nanomagnetic islands, specialized microscopy techniques can be useful for imaging the magnetization direction. Three often encountered techniques are presented in this section, each with certain advantages and disadvantages.
\subsubsection{Magnetic Force Microscopy}
Magnetic Force Microscopy (MFM) is a form of scanning probe microscopy that can measure out-of-plane magnetic fields. For this, a cantilever with a magnetic tip is used which is scanned across the sample at a very low height. One must take care that this magnetic tip does not significantly influence the magnetization of the sample itself.~\cite{Probing_MagnetoOptics} Also, if this tip were to simply scan over the surface, both the magnetic forces as well as the atomic forces would be measured. As we want to determine only the magnetic field, the topography of the sample is first determined using conventional Atomic Force Microscopy.~\cite{PEEM, NML_Carlton} Once the topography is known, the magnetic tip can scan the sample while maintaining a constant distance above it using the known topography. This decouples the measurement of magnetic forces from the atomic forces and allows one to determine just the magnetic field. Important to note is that the force measured is only the out-of-plane component of the magnetic field, because the cantilever can only move vertically and is thus only sensitive to the vertical component of the magnetic field.~\cite{NML_Carlton} \par
Because there are strong out-of-plane components on either end of a nanomagnet, when imaged using MFM one such island will manifest itself in the output as the combination of a region with positive and a region with negative out-of-plane magnetic field, because the stray magnetic field of a dipole curls back on itself.~\cite{NML_Carlton} An example of an MFM image of a chain of nanomagnets is shown at the top of~\cref{fig:Intro_Imaging}. \par
A disadvantage of MFM is that it does not image the magnetization directly, but rather the stray out-of-plane magnetic fields, which can make the results more difficult to interpret. It is however a cheaper technique that requires less safety measures than X-ray based techniques, which are discussed in the next section.

\subsubsection{Photoemission Electron Microscopy using X-ray Magnetic Circular Dichroism}
Photoemission Electron Microscopy (PEEM) images secondary electrons emitted from the sample upon irradiated with X-rays. This kind of electron microscopy can achieve a high spatial resolution of less than \SI{50}{\nano\metre}, with a typical probing depth in metals of about \SI{2}{\nano\metre}.~\cite{PEEM} The technique can be used to simply image the chemical and elemental structure of the sample, but can be adapted to image the magnetization direction in ferromagnets by utilizing an effect called X-ray Magnetic Circular Dichroism (XMCD). This is a physical phenomenon where the number of electrons emitted from the sample upon irradiation with circularly polarized X-rays depends on the magnetization direction of the surface of the sample.~\cite{NML_Carlton} More specifically, the quantity that is measured is the angle $\phi$ between the magnetization direction $\vb{m(\vb{r})}$ of the sample and the photon spin $\vec{\sigma}$, which is aligned with the photon propagation direction and changes sign when the photon helicity is reversed.~\cite{PEEM} The intensity of emitted electrons is given by
\begin{equation}
    I_{\mathrm{XMCD}} \propto M_{sat} \cos(\phi) \mathrm{.}
    \label{eq:XMCD}
\end{equation}
This way, PEEM can directly measure the in-plane magnetization direction, which makes it easier to interpret than an MFM image. A slightly more detailed description of the geometry and image acquisition of this kind of electron microscope can be found in~\cite{PEEM}. % TODO: QUESTION: Is this an expensive method? I suppose MFM is relatively cheap, but this requires a synchrotron which many papers do at the so-called 'Swiss Light Source', which I suppose is one of the few places where they have this equipment?
\begin{figure}
     \centering
     \begin{subfigure}[b]{0.8\textwidth}
         \centering
         \includegraphics[width=\textwidth]{Figures/Introduction/NML_Carlton - Figure 1.15 cropped.png}
     \end{subfigure}
     \begin{subfigure}[b]{0.8\textwidth}
         \centering
         \includegraphics[width=\textwidth]{Figures/Introduction/NML_Carlton - Figure 1.17 cropped.png}
     \end{subfigure}
     \caption{Typical MFM (top) and PEEM (bottom) image of the same chain of elliptical nanomagnets. Figures taken from \cite{NML_Carlton}.}
     \label{fig:Intro_Imaging}
\end{figure} % TODO: perhaps refer to the geometry of the chain as well, this was already mentioned before in the 'signal propagation' part so a quick reference should suffice

\subsubsection{Magneto-Optical Kerr Effect}
The Magneto-Optical Kerr Effect (MOKE) can also be used to determine the magnetization, albeit the average over a larger area on the order of several tens of \SI{}{\micro\metre\squared}. A linearly polarized laser beam is focused onto the sample, and the polarization state of the reflected light is measured in order to access the longitudinal Kerr effect.~\cite{MQCA_RoomTemp} The longitudinal Kerr effect says that, depending on the angle between the magnetization and the incoming light, the reflected light will become elliptically polarized to a certain degree.~\cite{KerrFaraday_book} This effect is small, but the ellipticity is directly proportional to the cosine of the angle, which can be used to derive the magnetization direction. There also exists the transversal Kerr effect, but this is not often used as it results in a change in reflectivity, which is more difficult to detect. \par 
The low resolution of this technique is due to the diffraction limit of the laser used, and for structures smaller than this diffraction limit the signal strength diminishes.~\cite{Probing_MagnetoOptics} One must take care that the focused laser beam does not excessively heat up the sample, as this could cause the measurement to influence the magnetization of the sample. The temperature rise for a \SI{2.5}{\milli\watt} laser was however found to be negligible.~\cite{Probing_MagnetoOptics} This measurement technique can be easier or cheaper to set up than the X-ray technique, but the (very) low resolution does not allow the imaging of individual nanomagnets in complex structures.
% TODO: To define the features: Electron beam lithography~\cite{NML_Carlton}, but i dont think there is a lot to say about that

\section{Physics}
In a crystalline material, the atoms are evenly spaced. Each atom has a discrete magnetic moment $\vb{m_i}$ associated with it. The interaction of many magnetic moments can give rise to macroscopic effects. Since every magnetic moment interacts with every other magnetic moment, the underlying problem is therefore a discrete N-body problem. Unfortunately, such a problem quickly becomes very hard if not impossible to solve analytically for even small $N$, hence restricting analytical calculations to very small systems.~\cite{abert2013discrete} Due to the lack of an analytical solution, many particular problems can only be solved approximately in a numerical manner.~\cite{abert2013discrete} Even with numerical techniques, the computational power or time required to solve an N-body problem increases rapidly with $N$, due to the number of interactions. For this reason, a continuum theory was developed, called the micromagnetic theory. In this formalism, the magnetization is represented by the continuous magnetization field, denoted by $\vb{M}(\vb{r}) = M_{sat} \vb{m}(\vb{r})$. This is the magnetic moment per unit volume averaged over a small region of space, with $\vb{m}(\vb{r})$ a unit vector, and $M_{sat}$ called the saturation magnetization.~\cite{Gilbert1956}
The size of this small region is characterized by the exchange length $\lambda$, a formula for which is given by \cref{eq:Energy_ExchangeEnergy_ExchangeLength}. It should be much smaller than the size of a single magnetic domain, yet much larger than a crystal unit cell, so on the order of several nanometres. It is clear that this continuum approximation is only applicable if locally all discrete magnetic moments try to align themselves parallel to each other, i.e.
\begin{equation}
    \vb{m_i} \approx \vb{m_j}~~\mathrm{if}~~\abs{\vb{r_i} - \vb{r_j}} < \lambda \mathrm{,}
\end{equation}
as is for example the case in ferromagnets.~\cite{abert2013discrete} \par
In situations where this approximation holds, the continuum theory can provide a significant computational improvement, because one can now use numerical cells with typical dimensions on the order of $\lambda$, which significantly decreases the amount of variables compared to the original N-body problem, where every single atom had to be taken into account. The characteristic size of a simulation using micromagnetic theory is on the order of tens to hundreds of nanometres, using millions of cells with dimensions on the order of $\lambda$.~\cite{abert2013discrete} Thus, the micromagnetic theory works well on a macroscopic scale and can be solved numerically in a reasonable amount of time, which would be wholly impossible with an N-body approach.


\subsection{Energy contributions}
The energy corresponding to a certain continuous magnetization field $\vb{m}(\vb{r})$ is the sum of several different contributions
\begin{equation}
    E = E_{exch} + E_{anis} + E_{demag} + E_{Zeeman} + E_{me} \mathrm{.} \label{eq:Energy_Terms}
\end{equation}
The physical nature of these different terms along with small derivations to accomodate them to the continuum approximation are presented in this section.
\subsubsection{Exchange energy}
\label{par:Energy_Exchange}
The exchange energy is of quantum mechanical origin. It tries to align neighboring spins and takes on the simple form
\begin{equation}
    E_{i,j} = -J \vb{S_i} \vdot \vb{S_j} \mathrm{.}
    \label{eq:Energy_ExchangeEnergy_Discrete}
\end{equation}
Summing over all contributions gives
\begin{equation}
    E = -\sum_{i,j} J_{i,j} S^2 \vb{n_i} \vdot \vb{n_j} \mathrm{,}
    \label{eq:Energy_ExchangeEnergy_SumDiscrete}
\end{equation}
with $\vb{S_i} = S \vb{n_i}$ and $\abs{n_i} = 1$. 
The sign of $J$ determines whether the spins align parallel ($J>0$) or anti-parallel ($J<0$). A parallel alignment results in a so-called ferromagnetic material, while an anti-parallel alignment is characteristic of an anti-ferromagnetic material. \par
Micromagnetic theory makes use of the magnetization field $\vb{m}(\vb{r})$, while \cref{eq:Energy_ExchangeEnergy_Discrete} is discrete. Two particles $i$ and $j$ only feel the exchange interaction when they are close to each other. This allows us to expand $\vb{m}(\vb{r}) \cdot \vb{m}(\vb{r} + \Delta\vb{r})$ in first order~\cite{abert2013discrete} to
\begin{align*}
    \vb{m}(\vb{r}) \cdot \vb{m}(\vb{r} + \Delta\vb{r}) &= 1 - \frac{1}{2}(\vb{m}(\vb{r}) - \vb{m}(\vb{r} + \Delta\vb{r}))^2 \\
    &\approx 1 - \frac{1}{2}\sum_i(\Delta\vb{r} \cdot \gradient{m_i})^2 \numberthis \label{eq:Energy_ExchangeEnergy_DotApprox} \mathrm{.}
\end{align*}
We can now write an equation similar to \cref{eq:Energy_ExchangeEnergy_SumDiscrete}, but for $\vb{m}(\vb{r})$:
\begin{equation*}
    E_{exch} = \int_\Omega \sum_i A_i \vb{m}(\vb{r}) \cdot \vb{m}(\vb{r} + \Delta\vb{r}_i) \mathrm{,}
\end{equation*}
which by substitution of \cref{eq:Energy_ExchangeEnergy_DotApprox} finally yields~\cite{abert2013discrete,Gilbert1956}
\begin{equation}
    E_{exch} = A_{ex} \int_\Omega \sum_i (\gradient{m_i}(\vb{r}))^2 \, d\vb{r} \mathrm{.} \label{eq:Energy_Term_Exchange}
\end{equation}
$A_{ex}$ is called the exchange stiffness constant.~\cite{Gilbert1956} The physical interpretation of this energy term is that the magnetization tries to align itself as smoothly as possible.
The exchange length $\lambda$ can now be calculated as
\begin{equation}
    \lambda = \sqrt{\frac{2 A_{ex}}{\mu_0 M_{sat}^2}} \mathrm{.}
    \label{eq:Energy_ExchangeEnergy_ExchangeLength}
\end{equation} % TODO: find a reference which mentions this (mumax3, ExchangeLength)
% TODO: mention that the exchange length is just a rule of thumb (see pdf of ExchangeLength reference)

\subsubsection{Anisotropy energy}
A material may exhibit anisotropy due to its crystal structure. The most often encountered types of anisotropy are uniaxial and cubic anisotropy. Uniaxial anisotropy is common in hexagonal or tetragonal crystal structures, while cubic anisotropy is often present in FCC or BCC structures.~\cite{Gilbert1956, abert2013discrete} \par
In the uniaxial case, the energy is minimal when the magnetization lies along a certain axis $\vb{u}$. In the cubic case, there are three such axes $\vb{e_i}, i=1,\dots,3$, which are all equivalent. The direction of $\vb{M}$ along this axis (i.e. $\vb{m}=\pm \vb{u}$) does not matter, such that $E(\vb{m}_{min}) = E(-\vb{m}_{min})$. For this reason, only even orders in the Taylor expansion are considered.~\cite{abert2013discrete} For uniaxial anisotropy this gives
\begin{equation}
    E_{anis,u} = - \int_\Omega \big(K_{u1} (\vb{m} \cdot \vb{u})^2 + K_{u2} (\vb{m} \cdot \vb{u})^4 + \dots\big) \, d\vb{r} \mathrm{.} \label{eq:Energy_Term_AnisUniaxial}
\end{equation}
For cubic anisotropy, one can use a similar symmetry reasoning~\cite{abert2013discrete} to find
\begin{equation}
    E_{anis,c} = \int_\Omega \big(K_{c1} (m_1^2m_2^2 + m_2^2m_3^2 + m_3^2m_1^2) + K_{c2} m_1^2m_2^2m_3^2\big) \, d\vb{r} \mathrm{,} \label{eq:Energy_Term_AnisCubic}
\end{equation}
with $m_i(\vb{r}) = \vb{m}(\vb{r}) \cdot \vb{e_i}, i=1,\dots,3$.

\subsubsection{Demagnetization energy/Magnetostatic energy}
The demagnetization energy, often also called the magnetostatic energy, is the energy arising from the interaction of every magnetic moment in a ferromagnetic material with the force of every other magnetic moment in the material.~\cite{NML_Carlton} In other words, it is the energy of the magnetization in the magnetic field created by itself.~\cite{abert2013discrete}
From classical electrodynamics it is known that, in the absence of currents,
\begin{align}
	\div{\vb{B}} &= 0 \label{eq:Energy_Demag_DivB0} \\
	\curl{\vb{H}} &= 0 \label{eq:Energy_Demag_CurlH0} \\
	\vb{B} &= \mu_0 (\vb{H} + \vb{M}) \label{eq:Energy_Demag_BHM} \mathrm{.}
\end{align}
From \cref{eq:Energy_Demag_CurlH0} immediately follows that $\exists \psi(\vb{r}): \vb{H} = -\gradient{\psi}$. By substituting \cref{eq:Energy_Demag_BHM} in \cref{eq:Energy_Demag_DivB0} it is then clear that
\begin{equation}
    \Delta \phi = \div{\vb{M}} \mathrm{.}
\end{equation}
The solution of this Laplace equation, for the boundary condition $\abs{\vb{r}} \rightarrow 0 \Rightarrow \psi \rightarrow 0$, is found using Green's function and Green's theorem:
\begin{align*}
    \psi(\vb{r}) &= -\frac{1}{4\pi}\int_\Omega \frac{\boldsymbol{\nabla'\,\vdot}\,\vb{M}(\vb{r'})}{\abs{\vb{r}-\vb{r'}}} \,d\vb{r'} + \frac{1}{4\pi}\int_{\partial\Omega} \frac{\vb{M}(\vb{r'}) \vdot \vb{n}}{\abs{\vb{r}-\vb{r'}}} \,d\vb{s'} \\
    &= \frac{1}{4\pi}\int_\Omega \vb{M}(\vb{r'}) \vdot \boldsymbol{\nabla'} \frac{1}{\abs{\vb{r}-\vb{r'}}} \,d\vb{r'} \mathrm{.}
\end{align*}
A more rigorous derivation is given in~\cite{abert2013discrete}.
From this explicit expression for $\psi(\vb{r})$, one can determine $\vb{H_{demag}} = -\gradient{\psi}$. The energy can then also be found through classical electrodynamics as
\begin{equation}
    E_{demag} = -\frac{\mu_0}{2} \int_\Omega \vb{M} \cdot \vb{H_{demag}} \, d\vb{r} \mathrm{.} \label{eq:Energy_Term_Demag}
\end{equation}
There is an additional $\frac{1}{2}$ factor because each interaction is counted twice.

\subsubsection{Zeeman energy}
If an external field is applied, an additional energy term similar to the demagnetization energy appears, but now without the factor $\frac{1}{2}$ because the interaction comes from an external source:
\begin{equation}
    E_{Zeeman} = -\mu_0 \int_\Omega \vb{M} \vdot \vb{H_{ext}} \, d\vb{r} \mathrm{.} \label{eq:Energy_Term_Zeeman}
\end{equation}

\subsubsection{Magneto-elastic energy}
\label{par:Energy_MagnetoElastic}
In case there is stress present in the crystal lattice, an additional magneto-elastic energy term can be formulated~\cite{Gilbert1956}. For an isotropic material this is given by
\begin{equation}
    E_{me} = B \int_\Omega \sum_{i,j} m_i m_j \frac{\partial s_i}{\partial x_j} \,d\vb{r} \mathrm{,}
\end{equation}
with $\vb{s}(\vb{r})$ the displacement field of the lattice.

\subsubsection{Other energy terms}
Other energy terms still exist to describe other specific phenomena, though these will not be used in this work. An example of such an energy term is the Dzyaloshinskii-Moriya interaction~\cite{DzyaloshinskiiMoriya}, which tries to orient neighboring spins perpendicular to each other, and is responsible for the stabilisation of skyrmions as described in~\cite{skyrmions}. \par Additional torques can also be taken into account, like the Zhang-Li or Slonczewski spin-transfer torques, which describe the interaction between a spin-polarised current and the magnetization~\cite{ZhangLiSpinTransferTorque, MuMax3, syllabus_PoAEaPD}. This effect can be used in information storage technologies like MRAM, magnetic field sensors, among many other applications.~\cite{syllabus_PoAEaPD}

\subsection{Landau-Lifschitz-Gilbert equation}
% TODO: Sort out the difference between LL eq. and LLG eq.
The aforementioned energy terms can be used to define the potential energy of a magnetization state $\vb{m}(\vb{r})$. Because $\vb{m}(\vb{r})$ is always a unit vector, its motion can be described by that of a rotating body, thus allowing the use of Lagrangian formalism with the kinetic energy term $-M_{sat}/\gamma \dot{\phi} \cos(\theta)$ to determine the dynamics of the system.~\cite{abert2013discrete} Solving this Lagrangian problem yields the Landau-Lifschitz (LL) equation. This is a partial differential equation given by~\cite{phd_leliaert}
\begin{equation}
	\frac{\partial \vb{m}}{\partial t} = - \gamma_0 \vb{m} \cross \vb{H_{eff}} - \lambda \vb{m} \cross (\vb{m} \cross \vb{H_{eff}}) \mathrm{.}
	\label{eq:LL}
\end{equation}
% mumax3-workshop says
% \begin{equation}
% 	\frac{\partial \vb{m}}{\partial t} = - \frac{\gamma}{1+\alpha^2} \big( \vb{m} \cross \vb{H_{eff}} - \alpha \vb{m} \cross (\vb{m} \cross \vb{H_{eff}}) \big) \mathrm{.}
% \end{equation}
In this equation, the so-called effective field $\vb{H_{eff}}$ is used, which is defined as
\begin{equation}
	\vb{H_{eff}} = - \frac{1}{\mu_0} \frac{\partial E}{\partial \vb{m}} \mathrm{,}
	\label{eq:H_eff}
\end{equation}
with the energy $E$ the sum of several contributions as explained before. $\partial E/\partial \vb{M}$ means the vector whose components are $\partial E/\partial M_x$, etc.~\cite{ThermFluc_SingleDomain} 
The term $\vb{M} \cross \vb{H_{eff}}$ causes the magnetic moment to precess around the effective field. The constant $\gamma_0 = \mu_0 \frac{ge}{2m_e}$, with $g$ the Landé factor, is related to the gyromagnetic ratio and determines the precession frequency $f=\frac{\gamma_0}{2\pi}\abs{\vb{H_{eff}}}$, then called the Larmor frequency.~\cite{phd_leliaert} If only this first term were present, the precession would occur forever, so Landau and Lifschitz included a phenomenological damping term $\vb{M} \cross (\vb{M} \cross \vb{H_{eff}})$ as well, which causes the moment $\vb{M}$ to slowly damp toward the effective field vector $\vb{H_{eff}}$.~\cite{NML_Carlton} The physical origins of this damping include eddy currents~\cite{phd_leliaert}, phonon excitation due to spin-lattice coupling~\cite{phd_leliaert}, %TODO: physical origins of damping

In order to obtain a physically more intuitive damping term, Gilbert \cite{Gilbert1955anomalous} replaced it with a term dependent on the time derivative of the magnetization, thus forming the Landau-Lifschitz-Gilbert (LLG) equation~\cite{ThermFluc_SingleDomain, phd_leliaert, LEL-17b}:
\begin{equation}
	\frac{\partial \vb{m}}{\partial t} = - \gamma_0\vb{m} \cross \vb{H_{eff}} + \alpha \vb{m} \cross \frac{\partial \vb{m}}{\partial t} \mathrm{,}
	\label{eq:LLG}
\end{equation}
where $\alpha>0$ is a dimensionless damping constant. If this equation is solved for $\partial \vb{m}/\partial t$, it is of the same form as the LL equation \eqref{eq:LL}. These LL and LLG equations are hence equivalent, where a substitution of $\lambda$ and $\gamma_0$ in the LL equation by $\frac{\gamma_0 \alpha}{1+\alpha^2}$ and $\frac{\gamma_0}{1+\alpha^2}$, respectively, yields the LLG equation.~\cite{ThermFluc_SingleDomain,phd_leliaert} Remark that this substitution indicates that, in the LLG equation, the precession frequency is dependent on the damping, which means that the LL and LLG equations only represent the same physics in the limit of low damping. Since the LL equation treats the damping only phenomenologically, it will not describe the physics correctly for high damping, whereas the LLG equation will.~\cite{phd_leliaert}
%TODO: Perhaps prove that modulus is preserved and that energy always decreases (see abert2013discrete)
%TODO: The sort out what the factors in front should be because there are conflicting or at least confusing sources
%TODO: tell that permalloy has alpha=0.01

\begin{figure}
    \centering
    \includegraphics[width=0.9\columnwidth]{Figures/Introduction/abert2013discrete - Figure 2.2.pdf}
    \caption{Motion of a magnetic moment as described by the LLG equation. The motion can be divided into the (a) precessional and (b) damping components. (c) Resulting motion including both precession and damping. Figure taken from~\cite{abert2013discrete}.}
    \label{fig:LLG_motion_Heff}
\end{figure}

\subsubsection{Thermal fluctuations}
% TODO: remove the line in this paragraph that just says where the references come from
The literature in this section comes from the references \cite{LEL-17b}. \par
The LLG equation only takes into account the fundamental physics of the magnetization dynamics at zero temperature. However, in our universe things exist at nonzero temperatures, and thermal fluctuations play an important role in magnetic logic devices. % TODO: when I know what the role is exactly, come back here and tell something small about it here
In order to simulate nanomagnetic structures at nonzero temperatures, Brown~\cite{ThermFluc_SingleDomain} developed a theory to model thermal fluctuations in single-domain particles, by adding a stochastic thermal field $\vb{H_{therm}}$ to the effective field $\vb{H_{eff}}$ in the LLG equation. This thermal field has to fulfill certain statistical properties, namely
\begin{align*}
    \langle \vb{H_{therm}} \rangle &= 0 \mathrm{,} \\
    \langle H_{therm,i}(t) H_{therm,j}(t') \rangle &= q \delta(t-t') \delta_{ij} \mathrm{,}
\end{align*}
where $q=(2 k_B T \alpha)/(M_{sat} \gamma V)$, with $V$ the volume of the single-domain particle. The angled brackets denote either a time average or correlation. With these random and uncorrelated properties of the time-dependent additional field term, the LLG equation becomes a Langevin equation.~\cite{ThermFluc_SingleDomain} However, a nanomagnetic island is not a perfect single-domain particle with uniform magnetization, and different nanomagnetic islands can influence each other through magnetostatic interactions.
Lyberatos~\cite{Lyberatos_1993} realized that every finite-difference cell in a numerical simulation can be seen as one such a single-domain particle, for which Brown's theory can then be applied.~\cite{phd_leliaert} More specifically, the thermal field was implemented in \mumax{}~\cite{LEL-17b,MuMax3} as
\begin{equation}
    \vb{H_{therm}} = \vec{\eta} \sqrt{\frac{2 \alpha k_B T}{M_{sat} \gamma V \Delta t}} \mathrm{,}
    \label{eq:H_therm}
\end{equation}
where $\vec{\eta}$ is a random vector, determined from a standard normal distribution, whose value is changed after every time step. This equation is such that the thermal fluctuations are independent of the spatial discretization (volume $V$ and time step $\Delta t$) used.
% Adaptive time step in mumax problem? Solvers?
There are many finite-difference solvers available in \mumax{}, for example different orders of Runge-Kutta solvers, some of which benefit from the first-same-as-last (FSAL) property. In such solvers, the last torque evaluation of the current step is the same as the first evaluation of the next step, which allows an increase in efficiency because this step only has to be evaluated once. However, the stochastic thermal field is not constant, and hence these solvers can no longer benefit from the FSAL property.~\cite{LEL-17b} Another complication is that the time step $\Delta t$ appears in the expression for $\vb{H_{therm}}$. Since some solvers in \mumax{} improve their efficiency by using an adaptive time step, one must take additional care that the thermal field is calculated correctly in such case. For a detailed description of the implementation of the stochastic thermal field in \mumax{}, we refer to~\cite{LEL-17b}.
% TODO: Can still include something on jump noise process and Curie temperature (phd_leliaert) and a bit more on solvers (start of appendix in phd_leliaert)
% TODO: Find a good place to talk about the advantages of running mumax on gpu, with parallelism etc. (see 'Micromagnetic simulations using Graphics Processing Units')

\section{Biaxial island}
\label{par:Biaxial_island}
The first part of this thesis consists of investigating the properties of a single biaxial island, as this will be the fundamental building block of more complex circuits. The geometry of such an island is one of the main factors determining the energy barrier between stable states. The geometry used in this thesis is the union of two perpendicular identical ellipses, as shown in \cref{fig:biaxial_island:geometryTypical}, where \SI{0}{\degree} is defined as the direction pointing to the right, as usual in mathematics. This geometry was chosen because it should be reasonably easy to manufacture due to the lack of sharp corners. This geometry has two `degrees of freedom', namely the long and short axis of the ellipse. We will define the roundness $\rho$ as the ratio of the short over the long axis of the ellipses, and the `overall size' $L$ as simply being equal to the long axis. For example, $(\rho, L)=(0.55, \SI{100}{\nano\metre})$ means that the long axis of each ellipse is \SI{100}{\nano\metre} while the short axis is \SI{55}{\nano\metre}.
\begin{figure}
    \centering
    \includegraphics[width=0.3\columnwidth]{Figures/biaxial_island/Geometry/geomPlus55.png}
    \caption{Typical geometry of the biaxial island under investigation, in this specific case for 55x\SI{100}{\nano\metre} ellipses, i.e. $(\rho, L)=(0.55, \SI{100}{\nano\metre})$. White represents ferromagnetic material, black is free space.}
    \label{fig:biaxial_island:geometryTypical}
\end{figure}

\subsection{Energy barrier}
Due to symmetry reasons, all of $\theta = k\frac{\pi}{4} , k\in\mathbb{Z}$ are equilibria. However, not all are stable; some are energy minima while some are energy maxima. To determine the energy barrier $E_{barrier}$ for random switching between stable states, the energy landscape as a function of magnetization angle must be calculated. If one were to simply set the magnetization $\vb{M}$ of the entire island in one specific direction and calculate the energy, one finds that the energy is the same for all such directions. This is due to the absence of the demagnetizing field caused by the shape anisotropy of the magnetic body.~\cite{Nonmonotonic_reversal} Thus, the magnetization should first be relaxed to accommodate for the shape of the island, which can be done using the \mumax{} command \code{minimize()}. However, this would result in a complete relaxation to an energy minimum, hence not yielding any useful information on the energy barrier which is the difference between an energy minimum and maximum. To circumvent this issue, an external magnetic field is applied to keep the average magnetization $\frac{1}{A}\int_A \vb{M}\,dA$ close to the initial direction. Practically, this `virtual' external field $B_{ext,v}$ is realized through the custom field functionality in \mumax{}, using \code{AddFieldTerm} and \code{AddEdensTerm}. \par
The procedure is therefore as follows. First the magnetization of the entire island $\vb{M}$ is initialized uniformly under an angle $\theta$ and an external magnetic field is applied along that same direction. Then, the magnetization is relaxed using \code{minimize()}. The internal energy, responsible for the energy landscape, is then equal to $E_{total} - E_{Zeeman}$. This procedure is repeated for different angles $\theta$ to generate an energy profile from which the energy barrier can be deduced.

\subsubsection{Optimal virtual external field magnitude}
The procedure as described above makes use of an external magnetic field. This is merely out of necessity to make sure that a magnetization initialized in an unstable maximum energy state stays there and does not relax to the energy minimum, which would prevent the determination of the height of the energy barrier. In reality, there is no such field, so this `virtual' external field should ideally be of small magnitude. To investigate the influence of this magnitude on the energy landscape, a simulation is carried out in which both the angle and the magnitude of the external magnetic field are varied. In \cref{fig:barrierLandscape-sweepBext} the relaxed magnetization angle $\theta$ and corresponding energy are plotted for different magnitudes of the external magnetic field $\abs{\vb{B_{ext,v}}}$, with the angle of this field taken at 64 equally spaced values from \SIrange{0}{90}{\degree}. Note that the average relaxed magnetization angle is not necessarily equal to the external magnetic field angle, as the relaxation process will tend to more or less cant the magnetization towards the closest intrinsic energy minimum.
\begin{figure}
    \centering
    \includegraphics[width=0.9\textwidth]{Figures/biaxial_island/BarrierLandscape/Plus_65_B25-0.001-div4_a128Pi_plotOptimized.pdf}
    \caption{Energy landscape between \SIrange{0}{90}{\degree} for various external magnetic field magnitudes $\abs{\vb{B_{ext,v}}}$ as shown in the legend. The external field angle was varied uniformly from \SIrange{0}{90}{\degree} in 64 steps for each magnitude, and each dot corresponds to one such angle. The horizontal axis denotes the average angle of the magnetization over the whole island after relaxation. Note that this is not necessarily equal to the external field angle, especially for low field magnitudes. The geometry used here is $(\rho, L)=(0.65, \SI{100}{\nano\metre})$.}
    \label{fig:barrierLandscape-sweepBext}
\end{figure}

At very high magnetic fields, the magnetization of the entire island will be nearly uniform and will not follow the shape of the island along the edges. As mentioned before, this causes the energy to be the same for all angles, since it is the relaxation of the magnetization in a non-uniform way which causes the anisotropy, due to the absence of the demagnetizing field.~\cite{Nonmonotonic_reversal} This is the case in the figure for the extremely high field $\abs{\vb{B_{ext,v}}}=\SI{25}{\tesla}$. \par
The lower the magnetic field, the more the magnetization relaxes to the minimum, as evident from the spacing of the dots for each external field magnitude. This is to be expected, as lower magnetic fields will have more difficulty keeping the magnetization pointed in the same direction, against the shape anisotropy. This could cause a problem when trying to determine the value of the unstable energy maximum. This is however not the case, as evident from the single data-point at \SI{45}{\degree}, which stays at the equilibrium even after relaxation. \par
When initializing the magnetization and external field at an angle of \SI{45}{\degree}, the average magnetization angle of the island stays at this unstable maximum after relaxation, for all examined field magnitudes. Since the energy barrier is simply the difference between this unstable maximum and the stable minimum, a low external field of \SI{1}{\milli\tesla} is enough if one is solely interested in the barrier height, and not the whole energy landscape. In case one is interested in the whole energy landscape, a field of about \SI{0.1}{\tesla} is needed as seen in the figure. This does, however, come at the cost of a less accurate calculation of the energy barrier itself. \par

\subsubsection{Energy barrier as function of geometry}
The geometry is the most important factor determining the energy barrier. Shown in \cref{fig:barrier-cell_size} is the dependence of the energy barrier on the roundness $\rho$, for different total sizes $L$. \par 
The behavior as a function of the roundness $\rho$ is most clear by only looking at the blue curve of this figure, i.e. for $L=\SI{128}{\nano\metre}$. Interesting is that the sign of the energy barrier is dependent on $\rho$. The energy barrier is calculated as $E_{barrier} = E(\theta=\pi/4) - E(\theta=0)$, so a positive value corresponds to the long axes of the ellipses being the easy axes, while a negative value indicates that an average diagonal magnetization at \SI{45}{\degree} is favored. For $\rho \approx 0.48$, the energy barrier is zero.
% TODO: change this explanation to: In general, the magnetization likes to be horizontal and vertical as these are the longest directions. However, as the 'lobes' become thinner, it is no longer energetically favorable for the whole island to be magnetized nearly uniformly, as this becomes problematic in at least two of the lobes.
This can be understood by looking at the detailed magnetization pattern corresponding to \SI{0}{\degree} and \SI{45}{\degree} in both regimes as shown in \cref{fig:barrier-magnetization}. For the high-roundness shapes as for example in the top row of the figure, the geometry resembles a square. It is known that the stable directions of a square are its diagonals as these are the longest directions of the geometry, and hence also the stable directions of a high-roundness geometry are the long axes of its constituent ellipses. The low-roundness shapes are more clearly consisting of two long ellipses. As explained in the introduction, the magnetization preferably orients itself along the length of such an ellipse. Hence, the sum of the stable states of two orthogonal ellipses is on average a diagonal magnetization. mentioned in the introduction, the magnetization will tend to align itself along the longest axis of a given geometry. Thus, it is logical that the situation in the top left of \cref{fig:barrier-magnetization} is not stable while the one on the right is. In conclusion, the magnetization preferably lies horizontally or vertically at any point in the island regardless of the roundness, though the shape of the island causes the average of these horizontal and vertical parts to either be horizontal and vertical itself, or diagonal.
\begin{figure}
     \centering
     \begin{subfigure}[b]{0.2\textwidth}
         \centering
         \includegraphics[width=\textwidth]{Figures/biaxial_island/BarrierMagnetization/mPlus_roundness0.60_a0.00.png}
         \caption*{Stable}
         \label{fig:barrier-magnetization-60x100_ortho}
     \end{subfigure}
     \begin{subfigure}[b]{0.2\textwidth}
         \centering
         \includegraphics[width=\textwidth]{Figures/biaxial_island/BarrierMagnetization/mPlus_roundness0.60_a0.79.png}
         \caption*{Unstable}
         \label{fig:barrier-magnetization-60x100_diag}
     \end{subfigure}
     \vskip1em
     \begin{subfigure}[b]{0.2\textwidth}
         \centering
         \includegraphics[width=\textwidth]{Figures/biaxial_island/BarrierMagnetization/mPlus_roundness0.20_a0.00.png}
         \caption*{Unstable}
         \label{fig:barrier-magnetization-20x100_ortho}
     \end{subfigure}
     \begin{subfigure}[b]{0.2\textwidth}
         \centering
         \includegraphics[width=\textwidth]{Figures/biaxial_island/BarrierMagnetization/mPlus_roundness0.20_a0.79.png}
         \caption*{Stable}
         \label{fig:barrier-magnetization-20x100_diag}
     \end{subfigure}
        \caption{Relaxed magnetization for two different geometries with $L=\SI{100}{\nano\metre}$. Top: $\rho=0.6$. Bottom: $\rho=0.2$. Left: average magnetization angle \SI{0}{\degree}. Right: \SI{45}{\degree}.}
        \label{fig:barrier-magnetization}
\end{figure}
The energy barrier also depends on the total size $L$; when halving $L$, the energy barrier becomes approximately 4 times smaller. This indicates that the energy barrier is proportional to the volume of the island, as its thickness was kept constant.

The energy barrier goes to zero for $\rho \rightarrow 1$, as one would expect for a perfectly round geometry. However, due to numerical error, the barrier is not calculated as exactly zero, which will be discussed in the next section.

\subsubsection{Numerical error: influence of cell size on the energy barrier}
When performing more resource-intensive simulations, it is advantageous to use as large a cell size as possible in order to minimize the total amount of cells in the simulation. Increasing the size of the cells does however also increase the numerical error originating from the discretization of the grid. The size of this numerical error was examined by determining the energy barrier for different shapes (varying roundness and overall size) and for different cell sizes of \SI{4}{\nano\metre}, \SI{2}{\nano\metre} and \SI{1}{\nano\metre}. The results of this calculation are shown in \cref{fig:barrier-cell_size}.
\begin{figure}
    \centering
    \includegraphics[width=0.9\columnwidth]{Figures/biaxial_island/Barrier/Plus_32-128_0.1-1_aPi4_B0.001_cell1nm.pdf}
    \caption{Energy barrier as a function of roundness $\rho$, for different total sizes $L$ as listed in the legend. Cell size \SI{1}{\nano\metre}.}
    \label{fig:barrier-cell_size}
\end{figure}
\begin{figure}
    \centering
    \includegraphics[width=0.9\columnwidth]{Figures/biaxial_island/Barrier/Plus_100_0.1-1_aPi4_B0.001.pdf}
    \caption{Energy barrier as a function of roundness $\rho$, for different cell sizes as listed in the legend. Long ellipse axis $L=\SI{100}{\nano\metre}$.}
    \label{fig:barrier-cell_size-100nm}
\end{figure}
The figure obtained with cells of \SI{1}{\nano\metre} is quite smooth, so it is justified that this size was used in the previous paragraphs. For \SI{2}{\nano\metre} cells, the curve becomes rougher, but the overall shape remains very similar to that of \SI{1}{\nano\metre} cells. Cells of \SI{4}{\nano\metre}, however, yield a very rough and almost stair-like energy landscape, which clearly indicates that this cell size is too large. \par
It is clear that in reality there should be no preferential direction for a round geometry. Because of the cartesian grid, however, a preferential orientation is created in the simulation. For large cells, the energy barrier for a round geometry becomes very large.
The smaller the grid, the more the simulation will resemble reality.
Everything taken into account, it seems that the optimal compromise between simulation speed and accuracy, is a \SI{2}{\nano\metre} cell size. However, if one is solely interested in the behavior corresponding to a specific value of the energy barrier, any cell size can be used as long as the corresponding energy barrier for that discretization is known.

\subsubsection{Energy landscape in presence of an external magnetic field}
% TODO: external field but there is not a lot of interesting things to say about this
The `virtual' external field $B_{ext,v}$ used in the previous paragraphs was solely there to make sure the magnetization remains in the energy maximum in order to determine the value of this maximum. Of course, it is also possible to put the biaxial island in a real external field, denoted with $B_{ext}$. This will make the energy term in \cref{eq:Energy_Term_Zeeman} nonzero, hence adding a cosine function to the energy landscape. In order to calculate the energy landscape, the virtual external field is still necessary to stabilize the magnetization in the desired direction, away from the global minimum. This can be done in \mumax{} by using a custom field for the virtual external field and using the normal external field functionality for the real external field, hence decoupling the calculation of their energies and allowing the virtual field to be ignored in the energy calculation. The energy landscape for a situation where the external field is applied at $3\pi/8$ is shown in \cref{fig:barrierLandscape_extField}.
% TODO: mention that there is a second external field
\begin{figure}
    \centering
    \includegraphics[width=0.9\columnwidth]{Figures/biaxial_island/BarrierLandscape/Ext_K0.1Ms2_Bext1e-2-1e-4_a3Pi8.pdf}
    \caption{Energy landscape as function of magnetization angle after relaxation, for different external field magnitudes $\abs{\vb{B_{ext}}}$ as listed in the legend, with the external field applied at an angle $3\pi/8$.}
    \label{fig:barrierLandscape_extField}
\end{figure}

\subsection{Thermal switching}
Real nanomagnets exist at nonzero temperatures. These thermal fluctuations cause the magnetization of every atom to feel an additional random torque. When these fluctuations randomly add up to a large enough torque, the magnetization can overcome the energy barrier and switch between two stable states. One of the main subjects of interest is the thermal switching frequency that accompanies a certain energy barrier.


\subsubsection{Theory}
Thermal switching is a random process, where the time between two switches is given by the relationship
\begin{equation}
    t_i = -\frac{1}{f_0} \exp(\frac{E_{barrier}}{k_B T}) \ln(1-P_i) \mathrm{,}
    \label{eq:Switching_time}
\end{equation}
valid in the regime where $E_{barrier} \gg k_B T$.~\cite{GYP-18} Here, $P_i$ is a uniformly random number between 0 and 1. The average value of $-\ln(1-P_i)$ is then $\ln(4)$, which makes the average time between two switches
\begin{equation}
    t_i = \frac{\ln(4)}{f_0} \exp(\frac{E_{barrier}}{k_B T})  \mathrm{.}
    \label{eq:Switching_time_average}
\end{equation}
Conversely, if one needs to achieve a specific switching time $t_i$, the required energy barrier is given by
\begin{equation}
    E_{barrier} = k_B T \ln(\frac{f_0 t_i}{\ln(4)}) \mathrm{.}
\end{equation}
The main unknown in all previous equations is the attempt frequency $f_0$. As this parameter is crucial to achieve a correct estimate for the switching time, it will be determined both theoretically and through numerical simulations on the biaxial geometry. Theoretical formulae are given in \cite{f0_alternative_Jonathan, MuMax3, LEL-17b}...
In the case of uniaxial anisotropy~\cite{MuMax3, LEL-17b, f0_mumax3_reference}, $f_0$ is given in [\SI{}{\radian\per\second}] by
\begin{equation}
    f_0 = \gamma_{LL} \frac{\alpha}{1+\alpha^2} \sqrt{\frac{8 K_{u1}^3 V}{2 \pi M_{sat}^2 k_B T}} \mathrm{,}
    \label{eq:f0_theoretical_uniaxial}
\end{equation}
with $\gamma_{LL}$ [\SI{}{\radian\per\tesla\per\second}] the gyromagnetic ratio, $\alpha$ the dimensionless damping parameter, $V$ the volume of the nanomagnet, $K_{u1}$ and $M_{sat}$ the uniaxial anisotropy constant and saturation magnetization as defined in the Physics section. The anisotropy constant in general is simply the energy barrier divided by the volume of the particle, so \cref{eq:f0_theoretical_uniaxial} can be adapted to a general anisotropic situation by substituting $K_{u1} \rightarrow E_{barrier}/V$.
An alternative formula can be found in \cite{f0_alternative_Jonathan}, which after simplification boils down to
\begin{equation}
    f_0 = \gamma_{LL} \frac{\alpha}{1+\alpha^2} \frac{4}{M_S V} \sqrt{\frac{E_{barrier}^3}{\pi k_B T}} \mathrm{.}
\end{equation}


%Idea: perhaps attempt frequency is the sinusoidal pattern in the long simulations, with period of about \SI{0.5}{\nano\second}.
% TODO: The gyromagnetic resonance frequency and response time (see ThermFluc_SingleDomain bottom of first page)
\subsubsection{Simulations}
From \eqref{eq:f0_theoretical_uniaxial}, it can be seen that the switching frequency depends on the damping constant $\alpha$, the temperature $T$, the energy barrier


\subsubsection{Results}
After running for \SI{1}{\micro\second} at \SI{300}{\kelvin}, with shape \SI{65}{\nano\metre} by \SI{100}{\nano\metre} ellipses in plus shape, the following angles were observed (\cref{fig:switching-alpha}):


\subsubsection{Counting switches}
Since the counting of switches is ambiguous, several different ways of counting them were performed:

1) Counting every crossing of a \SI{90}{\degree} line, e.g. a \SI{270}{\degree} switch is counted as 3 individual switches.

2) Counting every visible switch, with switches larger than \SI{90}{\degree} counted as 1 switch.

3) Looking every \SI{5}{\nano\second} and seeing if the magnetization direction changed without looking at the detailed magnetization in between those timestamps. This excludes 'spikes'.

Formula for $f_0$ derived from \eqref{eq:Switching_time}:
\begin{equation*}
    f_0 = \frac{N \ln(4)}{\Delta t} \exp(\frac{E_{barrier}}{k_B T})
\end{equation*}
with $\Delta t$ the simulation time and $N$ the number of switches.


I believe $K$, the anisotropy constant, is equal to $E_{barrier}/V$. Using this, I find a value of \SI{1e6}{\per\second}, which is not in line with \cref{tab:Switching_f0}.

Formula in \cite{MuMax3} works for the limit of a high barrier compared to the
thermal energy, so perhaps 49x100 is not very representative.

% TODO: This table is too wide, but currently serves as a sort of data storage
% TODO: The footnote is maybe quite long
The \cref{tab:Switching_f0} is for $\alpha=0.01$:
% TODO: the caption still has a python code in it
\ctable[
    cap = Switching Rates,
    caption = {Switching rates and corresponding $f_0$ values for different geometries and grid discretizations, all for $\alpha=0.01$.\\Python: \code{f0 = lambda N, dt, barrier, T: N*math.log(4)/dt*math.exp(barrier/1380649*16021766340/T)}},
    label = {tab:Switching_f0},
    pos = ht,
    sideways
    ]{
    c|c|c|c|c|c|c|c|c|c|c
    }{
    \tnote[a]{The switching rate ($\approx \SI{2}{\nano\second}$) is on the same timescale as the LLG dynamics, hence the exponential relationship of the Arrhenius law is no longer entirely applicable, causing the corresponding values for $f_0$ to be inaccurate. Furthermore, the energy barrier is no longer high compared to the thermal energy, which also reduces the accuracy of the Arrhenius law.}
    }{
        Geom [nm] & Barrier [meV] & Cell [nm] & T [K] & $\Delta t$ [ns] & (1) & (2) & (3) & $f_0$ (1) & $f_0$ (2) & $f_0$ (3) \\
        \hline
        49x100 & 24.7 \tmark[a] & 2 & 300 & 100 & 57 & 37 & 6 & \SI{2.06e9}{} & \SI{1.33e9}{} & \SI{2.16e8}{} \\
        \hline
        65x100 & 155.0 & 4 & 273 & 1000 & 13 & 13 & 9 & \SI{1.31e10}{} & \SI{1.31e10}{} & \SI{9.06e9}{} \\
        65x100 & 155.0 & 4 & 300 & 1000 & 15 & 11 & 7 & \SI{8.35e9}{} & \SI{6.12e9}{} & \SI{3.89e9}{} \\
        65x100 & 155.0 & 4 & 350 & 1000 & 51 & 41 & 30 & \SI{1.21e10}{} & \SI{9.69e9}{} & \SI{7.09e9}{} \\
        65x100 & 286.0 & 2 & 350 & 500 & 0 & 0 & 0 & 0 & 0 & 0 \\
        65x100 & 207.9 & 3.125 & 350 & 1000 & 12 & 9 & 7 & \SI{1.64e10}{} & \SI{1.23e10}{} & \SI{9.56e9}{} \\
        \hline
        100x100 & 241.1 & 4 & 300 & 1000 & 0 & 0 & 0 & 0 & 0 & 0 \\
        100x100 & 241.1 & 4 & 350 & 1000 & 5 & 5 & 5 & \SI{2.06e10}{} & \SI{2.06e10}{} & \SI{2.06e10}{} 
    }


\begin{figure}
     \centering
     \begin{subfigure}[b]{0.75\textwidth}
         \centering
         \includegraphics[width=\textwidth]{Figures/biaxial_island/Switching/65x100_300K_alpha0.1_1µs_4nm.pdf}
         \caption{$\alpha=0.1$}
         \label{fig:switching-alpha-0.1}
     \end{subfigure}
     \hfill
     \begin{subfigure}[b]{0.75\textwidth}
         \centering
         \includegraphics[width=\textwidth]{Figures/biaxial_island/Switching/65x100_300K_alpha0.01_1µs_4nm.pdf}
         \caption{$\alpha = 0.01$}
         \label{fig:switching-alpha-0.01}
     \end{subfigure}
        \caption{Switching for different $\alpha$, with grid discritization of \SI{4}{\nano\metre} at \SI{300}{\kelvin}, for 65x100.}
        \label{fig:switching-alpha}
\end{figure}

\begin{figure}
     \centering
     \begin{subfigure}[b]{0.49\textwidth}
         \centering
         \includegraphics[width=\textwidth]{Figures/biaxial_island/Switching/65x100_300K_alpha0.01_1µs_4nm.pdf}
         \caption{\SI{300}{\kelvin}, 65x100}
         \label{fig:switching-temp-300-65x100}
     \end{subfigure}
     \hfill
     \begin{subfigure}[b]{0.49\textwidth}
         \centering
         \includegraphics[width=\textwidth]{Figures/biaxial_island/Switching/65x100_350K_alpha0.01_1µs_4nm.pdf}
         \caption{\SI{350}{\kelvin}, 65x100}
         \label{fig:switching-temp-350-65x100}
     \end{subfigure}
     \begin{subfigure}[b]{0.49\textwidth}
         \centering
         \includegraphics[width=\textwidth]{Figures/biaxial_island/Switching/100x100_300K_alpha0.01_1µs_4nm.pdf}
         \caption{\SI{300}{\kelvin}, 100x100}
         \label{fig:switching-temp-300-100x100}
     \end{subfigure}
     \hfill
     \begin{subfigure}[b]{0.49\textwidth}
         \centering
         \includegraphics[width=\textwidth]{Figures/biaxial_island/Switching/100x100_350K_alpha0.01_1µs_4nm.pdf}
         \caption{\SI{350}{\kelvin}, 100x100}
         \label{fig:switching-temp-350-100x100}
     \end{subfigure}
    \caption{Switching for different temperatures, with grid discritization of \SI{4}{\nano\metre} and $\alpha = 0.01$.}
    \label{fig:switching-temp}
\end{figure}
\begin{figure}
    \centering
    \includegraphics[width=0.9\columnwidth]{Figures/biaxial_island/Switching/49x100_300K_alpha0.01_100ns_2nm.pdf}
    \caption{Switching for a long ellipse axis of \SI{49}{\nano\metre}, which has a very low energy barrier, with grid discretization of \SI{2}{\nano\metre} at \SI{300}{\kelvin} and $\alpha = 0.01$.}
    \label{fig:switching-49x100-300}
\end{figure}

\subsection{External field}
\begin{figure}
     \centering
     \begin{subfigure}[b]{0.65\textwidth}
         \centering
         \includegraphics[width=\textwidth]{Figures/biaxial_island/BarrierLandscape/Ext_K0.1Ms2_Bext1e-2-1e-4_aPi4.pdf}
         \caption{External angle $\pi/4$}
         \label{fig:barrierLandscape-externalAngle-pi4}
     \end{subfigure}
     \hfill
     \begin{subfigure}[b]{0.65\textwidth}
         \centering
         \includegraphics[width=\textwidth]{Figures/biaxial_island/BarrierLandscape/Ext_K0.1Ms2_Bext1e-2-1e-4_a3Pi8.pdf}
         \caption{External angle $3\pi/8$}
         \label{fig:barrierLandscape-externalAngle-3pi8}
     \end{subfigure}
     \begin{subfigure}[b]{0.65\textwidth}
         \centering
         \includegraphics[width=\textwidth]{Figures/biaxial_island/BarrierLandscape/Ext_K0.1Ms2_Bext1e-2-1e-4_aPi2.pdf}
         \caption{External angle $\pi/2$}
         \label{fig:barrierLandscape-externalAngle-pi2}
     \end{subfigure}
     \hfill
    \caption{Energy landscape for different external field angles (plots a, b, c), and different external field strengths (legend).}
    \label{fig:barrierLandscape-externalAngle}
\end{figure}

\section{Interaction between islands}
\subsection{Two islands}
The interaction can be decoupled in to the shape anisotropy energy and the interaction energy. Since we already determined the shape anisotropy in the previous part, the only thing left to do is determine the interaction energy. This can be a function of the distance between two particles, their size, and of course their inner magnetization angle.

\subsubsection{Interaction energy as function of magnetization angle}
Distance between the two particles is \SI{128}{\nano\metre}, their size is \SI{100}{\nano\metre} in this subsubsection. The effect of changing either of these two parameters will be examined in the next subsubsections.

Observations for islands without biaxial anisotropy ($\rho = 0.49$):
- There is one 'racetrack' in the energy landscape, it is basically all diagonal troughs and ridges
- The global minima are for magnetization in each other's direction (0 or 180 deg) which is understandable because the magnetization prefers to lay among the longest direction of a geometric structure
- The configuration (90 deg, 270 deg) is unstable and will like to relax to the global minimum (0, 0) or (180, 180).

Observations for islands with biaxial anisotropy ($\rho = 0.66$):
- Of course, an additional energy grid of sine functions with period 90 degrees is now added to the anisotropyless landscape, which creates local minima and maxima
- The configuration (90 deg, 270 deg) is now stable for a positive energy barrier

If one rotates one of the islands, then the corresponding 90 degree spaced lines (horizontal or vertical in the plot depending on which island is rotated) will move up down left or right in the plot.


\newpage
\bibliographystyle{IEEEtran}
\bibliography{bibliography/bibliography.bib}


\end{document}
