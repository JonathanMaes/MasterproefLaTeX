\documentclass[10pt,a4paper]{article}

%% PACKAGES WITH OPTIONS
\usepackage[english]{babel}
\usepackage[T1]{fontenc}
\usepackage[utf8]{inputenc}
\usepackage[version=4]{mhchem} % Chemistry and isotopes with \ce{} (other option: 'chemformula' \ch{})
\usepackage[numbers]{natbib} % numbers: needed argument, otherwise error
\usepackage[section]{placeins} % Place floats in their own section
\usepackage[nottoc,notlot,notlof]{tocbibind} % Include bibliography in table of contents

%% PACKAGES WITHOUT OPTIONS
\usepackage{adjustbox} % Use \begin{adjustbox}{center} environment for too wide floats
\usepackage{amsmath, amsfonts, amssymb}
\usepackage{enumitem} % Lists
\usepackage{flafter} % Place floats after their first appearance in text
\usepackage{float} % [H] modifier for floats etc.
\usepackage{graphicx}
\usepackage{hyperref, cleveref} % Creates links (boxes in various colors around words)
\usepackage{indentfirst} % Always indent the first line of a section
\usepackage{listings} % Typesetting source code
\usepackage{physics} % Vector calculus etc.
\usepackage{siunitx} % Fancy display of SI units
\usepackage{subcaption} % Enables subfigures
\usepackage{titling}
\usepackage{xcolor} % Anything with color e.g. \fcolorbox{black}{red}{}
\usepackage{xurl} % Loads package 'url' and breaks urls nicely

%% PACKAGES SETUP COMMANDS
\hypersetup{colorlinks=true,urlcolor=blue,citecolor=gray}

%% COMMANDS
% Bold and italic vector symbols (preferably use \vb{} instead)
\renewcommand{\vec}[1]{\boldsymbol{#1}}
% Monospaced inline code (for multiline code, use package 'listings')
\newcommand{\code}[1]{\texttt{#1}}
% Equals sign, with number above referencing some equation
\newcommand{\numeq}[1]{\stackrel{\scriptscriptstyle(\mkern-1.5mu#1\mkern-1.5mu)}{=}}
% If-and-only-if sign, with number above referencing some equation
\newcommand{\numiff}[1]{\stackrel{\scriptscriptstyle(\mkern-1.5mu#1\mkern-1.5mu)}{\Leftrightarrow}}
% Numbers a single line in a no-numbering multiline equation* or align*
\newcommand{\numberthis}{\addtocounter{equation}{1}\tag{\theequation}}
% MuMax3
\newcommand{\mumax}{$\mathsf{mumax}^3$}

%% TITLE VARIABLES
\author{Jonathan Maes}
\title{Biaxial nanomagnets as building block for balanced half-adders}


\begin{document}

\begin{titlingpage}
\maketitle
\end{titlingpage}

\newpage
\pagenumbering{roman}

\tableofcontents
\newpage
\pagenumbering{arabic}

\section{Introduction}
This thesis makes use of the \mumax{} micromagnetic simulation tool~\cite{MuMax3}.
Gilbert is \cite{Gilbert1956} and Lifdau is \cite{LANDAU1992}.

Interesting sources regarding the random thermal switching could be \cite{ThermFluc_SingleDomain, RandomSwitch_MonteCarlo, Nonmonotonic_reversal}, and perhaps \cite{MagDynamics_JumpNoise}.
Exchange bias is explained in \cite{ExchangeBias, ExchangeBias_nanostructures, ExchangeBias_Mechanisms}.
Quantum Cellular Automata (QCA) are discussed in \cite{QCA_Algorithms, QCA_GameOfLife}. Magnetic QCA (MQCA) are discussed in \cite{MQCA_MajorityGate, MQCA_RoomTemp, MQCA_ImageRecognition}.




\subsection{Domains}
In 1907, Weiss proposed the idea that a ferromagnetic material contains several uniformly magnetized domains, but that the magnetization of each domain individually is random, thus resulting in a macroscopic zero net magnetization of a large amount of material.~\cite{MuMax3_advances} The uniform alignment inside a single domain is due to the quantum mechanical Heisenberg-Dirac exchange interaction.~\cite{MuMax3_advances, heisenberg1928theorie} The magnetostatic energy makes it unfavorable for neighboring moments to align themselves parallel to one another, which competes with the exchange energy which tends to align neighboring moments in the same direction. Since the exchange energy only works on small length scales, domains are formed under the influence of the exchange energy, while different domains tend to cancel out their fields due to the magnetostatic energy. The typical size of these domains is on the order of several tens of micrometers. When the size of the material becomes on the order of micrometers, the domains form very symmetric patterns in order to minimize their energy, and on even smaller length scales only one domain remains.~\cite{NML_Carlton} So, if one makes a droplet of ferromagnetic material smaller than this typical size, the droplet will have a nearly uniform magnetization, with a magnitude equal to the saturation magnetization of the material.~\cite{NML_Carlton} Such a small droplet can for example be defined using electron beam lithography.~\cite{MQCA_RoomTemp, NML_Carlton} By introducing anisotropy in the droplet, which can be either magnetocrystalline or shape anisotropy, one or more axes can be made energetically favorable for the magnetization to align itself along. Magnetocrystalline anisotropy favors the crystal axes. Shape anisotropy can be realized by giving the droplet an elliptical shape instead of a perfect circle. In the uniaxial case, there is one stable magnetization axis, and the two directions `up' and `down' along this axis can be related to bits `0' and `1'.~\cite{MQCA_RoomTemp} This is not to be confused with unidirectional anisotropy, which only has one stable magnetization direction, and can be achieved using the exchange bias effect.~\cite{ExchangeBias_Mechanisms,ExchangeBias_nanostructures,ExchangeBias} A lot of research has been conducted to use uniaxial anisotropy for computation, and logic gates and wires have been proposed, which use classical magnetostatic interactions to propagate the information.~\cite{GYP-18,MQCA_MajorityGate,SwitchingForced_EnergyEfficient} The use of two favorable axes, i.e. biaxial anisotropy, allows for a higher logic density, as the four stable directions `up', `down', `right' and `left' can be related to `00', `01', `10' and `11'.~\cite{MQCA_ImageRecognition} This can occur for cubic crystals, and can also be realized by giving the shape of the droplet more symmetry, for example by making the shape the union of two ellipses.


\subsection{Quantum Cellular Automata}
Traditional digital logic technologies like CMOS use field-effect transistors to control the flow of electrons. This kind of architecture fundamentally only allows information to flow in one direction. Other architectures allowing communication between the output and input in both ways could allow certain computations to be executed faster. One approach to realizing such an architecture are the Quantum Cellular Automata (QCA), which use quantum effects, in a broad sense, to make logic gates. Such devices can be orders of magnitude smaller and more energy efficient than traditional CMOS technology.~\cite{MQCA_RoomTemp} There are several possibilities to realize QCA, which can generally be classified as either electronic or magnetic QCA. Electronic QCA (EQCA) make use of the forces between electrons, while magnetic QCA (MQCA) leverage the magnetic moments of atoms. EQCA is called ``quantum'' because it uses quantum mechanical tunneling of charge between quantum dots, MQCA are quantum because of the exchange interaction between individual atomic magnetic moments.~\cite{MQCA_RoomTemp} What all these QCA have in common, is that every fundamental building block (electrons, magnetic moments...) of these automata influences every other building block in the automaton, thus allowing the aforementioned two-way flow of information between what would traditionally be referred to as input and output. The MQCA are the main subject of this paper, but it is interesting to take a short look at EQCA as well, for both share some ideas and problems. \par
% Small segue to quantum dot cellular automata
One specific implementation of EQCA makes use of square cells, each with four quantum dots at the vertices of a square. These quantum dots can accomodate an electron, and each cell is made such that there are always two excess electrons present. It is then energetically favorable for these electrons to occupy two diagonal quantum dots.~\cite{QCA_DigitalLogicGate} The two diagonal configurations are then the `0' and `1' states. An extensively studied logic gate using this architecture is the three-input majority logic gate. It consists of five cells arranged in a plus-shape, with three inputs and one output. By fixing one of the inputs to 0 or 1, an AND or an OR operation can be realized, respectively. A drawback for these EQCA is that they require low electron temperatures in order to work reliably, as otherwise thermal smearing of the charge states of the dots becomes an issue.~\cite{QCA_DigitalLogicGate}


\subsection{Signal Propagation in uniaxial case}
% TODO: Uniaxial anisotropy: explain that it likes to be magnetized along the long axis (NML_Carlton p.9-10 and beyond can be a good introduction)

% TODO: Section on propagation of information along nanomagnet chians, for uniaxial and biaxial situation (seems like biaxial is less easy to find information about than uniaxial so need to find some more papers for that)
Interesting is the behavior of chains of nanomagnets, 
Piezoelectric
Spin transfer torque

\subsection{Imaging the magnetization}
- Magnetic Force Microscopy: images stray dipole fields~\cite{NML_Carlton}
- Photoemission Electron Microscopy: directly images the magnetization direction, relies on X-ray Magnetic Circular Dichroism~\cite{NML_Carlton}
- Magneto-optical measurement: a linearly polarized laser beam is focused onto the networks, and the polarization state of the reflected light is measured in order to access the longitudinal Kerr effect.~\cite{MQCA_RoomTemp}

- To define the features: Electron beam lithography~\cite{NML_Carlton}


\section{Physics}
% === THE FOLLOWING IS A NONCOMPREHENSIVE SUMMARY OF THE PHYSICS ===
The main formulae are derived in \cite{abert2013discrete} or \cite{NML_Carlton} and in the \mumax{} advances \cite{MuMax3_advances}. \par
In a crystalline material, the atoms are evenly spaced. Each atom has a discrete magnetic moment $\vb{m_i}$ associated with it. The interaction of many magnetic moments can give rise to macroscopic effects. Since every magnetic moment interacts with every other magnetic moment, the underlying problem is therefore a discrete N-body problem. Unfortunately, such a problem quickly becomes very hard if not impossible to solve analytically for even small $N$, hence restricting analytical calculations to very small systems.~\cite{abert2013discrete} Due to the lack of an analytical solution, many particular problems can only be solved approximately in a numerical manner.~\cite{abert2013discrete} Even with numerical techniques, the computational power or time required to solve an N-body problem increases rapidly with $N$, due to the number of interactions. For this reason, a continuum theory was developed, called the micromagnetic theory. In this formalism, the magnetization is represented by the continuous magnetization field, denoted by $\vb{M}(\vb{r}) = M_S \vb{m}(\vb{r})$. This is the magnetic moment per unit volume averaged over a small region of space, with $\vb{m}(\vb{r})$ a unit vector.~\cite{Gilbert1956}
The size of this small region is characterized by the exchange length $\lambda$. It should be much smaller than the size of a single magnetic domain, yet much larger than a crystal unit cell, so on the order of several nanometres. It is clear that this continuum approximation is only applicable if locally all discrete magnetic moments try to align themselves parallel to each other, i.e.
\begin{equation}
    \vb{m_i} \approx \vb{m_j}~~\mathrm{if}~~\abs{\vb{r_i} - \vb{r_j}} < \lambda \mathrm{,}
\end{equation}
as is for example the case in ferromagnets.~\cite{abert2013discrete} \par
In situations where this approximation holds, the continuum theory can provide a significant computational improvement, because one can now use numerical cells with typical dimensions on the order of $\lambda$, which significantly decreases the amount of variables compared to the original N-body problem, where every single atom had to be taken into account. The characteristic size of a simulation using micromagnetic theory is on the order of tens to hundreds of nanometres, using millions of cells with dimensions on the order of $\lambda$.~\cite{abert2013discrete} Thus, the micromagnetic theory works well on a macroscopic scale and can be solved numerically in a reasonable amount of time, which would be wholly impossible with an N-body approach.


\subsection{Energy contributions}
The energy corresponding to a certain continuous magnetization field $\vb{m}(\vb{r})$ is the sum of several different contributions
\begin{equation}
    E = E_{exch} + E_{anis} + E_{demag} + E_{Zeeman} \mathrm{.} \label{eq:Energy_Terms}
\end{equation}
The physical nature of these different terms along with small derivations to accomodate them to the continuum approximation are presented in this section.
\subsubsection{Exchange energy}
The exchange energy is of quantum mechanical origin. It tries to align neighboring spins and takes on the simple form
\begin{equation}
    E_{i,j} = -J \vb{S_i} \vdot \vb{S_j} \mathrm{.}
    \label{eq:Energy_ExchangeEnergy_Discrete}
\end{equation}
Summing over all contributions gives
\begin{equation}
    E = -\sum_{i,j} J_{i,j} S^2 \vb{n_i} \vdot \vb{n_j} \mathrm{,}
    \label{eq:Energy_ExchangeEnergy_SumDiscrete}
\end{equation}
with $\vb{S_i} = S \vb{n_i}$ and $\abs{n_i} = 1$. 
The sign of $J$ determines whether the spins align parallel ($J>0$) or anti-parallel ($J<0$). A parallel alignment results in a so-called ferromagnetic material, while an anti-parallel alignment is characteristic of an anti-ferromagnetic material.
% TODO: Exchange bias? Hier misschien niet op zijn plaats maar heeft wel te maken met FM/AFM

Micromagnetic theory makes use of the magnetization field $\vb{m}(\vb{r})$, while \cref{eq:Energy_ExchangeEnergy_Discrete} is discrete. Two particles $i$ and $j$ only feel the exchange interaction when they are close to each other. This allows us to expand $\vb{m}(\vb{r}) \cdot \vb{m}(\vb{r} + \Delta\vb{r})$ in first order~\cite{abert2013discrete} to
\begin{align*}
    \vb{m}(\vb{r}) \cdot \vb{m}(\vb{r} + \Delta\vb{r}) &= 1 - \frac{1}{2}(\vb{m}(\vb{r}) - \vb{m}(\vb{r} + \Delta\vb{r}))^2 \\
    &\approx 1 - \frac{1}{2}\sum_i(\Delta\vb{r} \cdot \gradient{m_i})^2 \numberthis \label{eq:Energy_ExchangeEnergy_DotApprox} \mathrm{.}
\end{align*}
We can now write an equation similar to \cref{eq:Energy_ExchangeEnergy_SumDiscrete}, but for $\vb{m}(\vb{r})$:
\begin{equation*}
    E_{exch} = \int_\Omega \sum_i A_i \vb{m}(\vb{r}) \cdot \vb{m}(\vb{r} + \Delta\vb{r}_i) \mathrm{,}
\end{equation*}
which by substitution of \cref{eq:Energy_ExchangeEnergy_DotApprox} finally yields~\cite{abert2013discrete,Gilbert1956}
\begin{equation}
    E_{exch} = A \int_\Omega \sum_i (\gradient{m_i}(\vb{r}))^2 \, d\vb{r} \mathrm{.} \label{eq:Energy_Term_Exchange}
\end{equation}
$A$ is called the exchange stiffness constant.~\cite{Gilbert1956}

\subsubsection{Anisotropy energy}
A material may exhibit anisotropy due to its crystal structure. The most often encountered types of anisotropy are uniaxial and cubic anisotropy. Uniaxial anisotropy is common in hexagonal or tetragonal crystal structures, while cubic anisotropy is often present in FCC or BCC structures.~\cite{Gilbert1956, abert2013discrete} \par
In the uniaxial case, the energy is minimal when the magnetization lies along a certain axis $\vb{u}$. In the cubic case, there are three such axes $\vb{e_i}, i=1,\dots,3$, which are all equivalent. The direction of $\vb{M}$ along this axis (i.e. $\vb{m}=\pm \vb{u}$) does not matter, such that $E(\vb{m}_{min}) = E(-\vb{m}_{min})$. For this reason, only even orders in the Taylor expansion are considered.~\cite{abert2013discrete} For uniaxial anisotropy this gives
\begin{equation}
    E_{anis,u} = - \int_\Omega \big(K_{u1} (\vb{m} \cdot \vb{u})^2 + K_{u2} (\vb{m} \cdot \vb{u})^4 + \dots\big) \, d\vb{r} \mathrm{.} \label{eq:Energy_Term_AnisUniaxial}
\end{equation}
For cubic anisotropy, one can use a similar symmetry reasoning~\cite{abert2013discrete} to find
\begin{equation}
    E_{anis,c} = \int_\Omega \big(K_{c1} (m_1^2m_2^2 + m_2^2m_3^2 + m_3^2m_1^2) + K_{c2} m_1^2m_2^2m_3^2\big) \, d\vb{r} \mathrm{,} \label{eq:Energy_Term_AnisCubic}
\end{equation}
with $m_i(\vb{r}) = \vb{m}(\vb{r}) \cdot \vb{e_i}, i=1,\dots,3$.

\subsubsection{Demagnetization energy/Magnetostatic energy}
The demagnetization energy, often also called the magnetostatic energy, is the energy arising from the interaction of every magnetic moment in a ferromagnetic material with the force of every other magnetic moment in the material.~\cite{NML_Carlton} In other words, it is the energy of the magnetization in the magnetic field created by itself.~\cite{abert2013discrete}
From classical electrodynamics it is known that, in the absence of currents,
\begin{align}
	\div{\vb{B}} &= 0 \label{eq:Energy_Demag_DivB0} \\
	\curl{\vb{H}} &= 0 \label{eq:Energy_Demag_CurlH0} \\
	\vb{B} &= \mu_0 (\vb{H} + \vb{M}) \label{eq:Energy_Demag_BHM} \mathrm{.}
\end{align}
From \cref{eq:Energy_Demag_CurlH0} immediately follows that $\exists \psi(\vb{r}): \vb{H} = -\gradient{\psi}$. By substituting \cref{eq:Energy_Demag_BHM} in \cref{eq:Energy_Demag_DivB0} it is then clear that
\begin{equation}
    \Delta \phi = \div{\vb{M}} \mathrm{.}
\end{equation}
The solution of this Laplace equation, for the boundary condition $\abs{\vb{r}} \rightarrow 0 \Rightarrow \psi \rightarrow 0$, is found using Green's function and Green's theorem:
\begin{align*}
    \psi(\vb{r}) &= -\frac{1}{4\pi}\int_\Omega \frac{\boldsymbol{\nabla'\,\vdot}\,\vb{M}(\vb{r'})}{\abs{\vb{r}-\vb{r'}}} \,d\vb{r'} + \frac{1}{4\pi}\int_{\partial\Omega} \frac{\vb{M}(\vb{r'}) \vdot \vb{n}}{\abs{\vb{r}-\vb{r'}}} \,d\vb{s'} \\
    &= \frac{1}{4\pi}\int_\Omega \vb{M}(\vb{r'}) \vdot \boldsymbol{\nabla'} \frac{1}{\abs{\vb{r}-\vb{r'}}} \,d\vb{r'} \mathrm{.}
\end{align*}
A more rigorous derivation is given in~\cite{abert2013discrete}.
From this explicit expression for $\psi(\vb{r})$, one can determine $\vb{H_{demag}} = -\gradient{\psi}$. The energy can then also be found through classical electrodynamics as
\begin{equation}
    E_{demag} = -\frac{\mu_0}{2} \int_\Omega \vb{M} \cdot \vb{H_{demag}} \, d\vb{r} \mathrm{.} \label{eq:Energy_Term_Demag}
\end{equation}
There is an additional $\frac{1}{2}$ factor because each interaction is counted twice.

\subsubsection{Zeeman energy}
If an external field is applied, an additional energy term similar to the demagnetization energy appears, but now without the factor $\frac{1}{2}$ because the interaction comes from an external source:
\begin{equation}
    E_{Zeeman} = -\mu_0 \int_\Omega \vb{M} \vdot \vb{H_{ext}} \, d\vb{r} \mathrm{.} \label{eq:Energy_Term_Zeeman}
\end{equation}

% TODO: Gilbert also mentions a magnetoelastic term

\subsection{Landau-Lifschitz-Gilbert equation}
% TODO: Sort out the difference between LL eq. and LLG eq.
The Landau-Lifschitz(-Gilbert) equation is a partial differential equation given by
\begin{equation}
	\frac{\partial \vb{M}}{\partial t} = - \gamma_0 \vb{M} \cross \vb{H_{eff}} - \lambda \vb{M} \cross (\vb{M} \cross \vb{H_{eff}}) \mathrm{.}
\end{equation}
PDE requires an effective field $\vb{H_{eff}}$ which is defined as
\begin{equation}
	\vb{H_{eff}} = - \frac{1}{\mu_0} \frac{\partial E}{\partial \vb{M}}
\end{equation}
the energy $E$ is the sum of several contributions, all explained in the next section.
The term $\vb{M} \cross \vb{H_{eff}}$ causes the magnetic moment to precess around effective field. If only this term were present, the precession would occur forever, so a phenomenological damping term $\vb{M} \cross (\vb{M} \cross \vb{H_{eff}})$ was included as well, which causes the moment $\vb{M}$ to slowly damp toward the effective field vector $\vb{H_{eff}}$.~\cite{NML_Carlton}

Gilbert \cite{Gilbert1956} replaced the damping term with a term dependent on the time derivative of the magnetization:
\begin{equation}
	\frac{\partial \vb{M}}{\partial t} = - \gamma_0 \Big( \vb{M} \cross \vb{H_{eff}} - \lambda \vb{M} \cross (\vb{M} \cross \vb{H_{eff}}) \Big) \mathrm{.}
\end{equation}

\section{Literature}
- Biaxial signal propagation? Very different from uniaxial (NML)


\section{Tests}
\subsection{Biaxial island}
The first part of this thesis consists of investigating the properties of a single biaxial island, as this will be the fundamental building block of more complex circuits. The geometry of such an island is one of the main factors determining the energy barrier between stable states. The geometry under investigation here consists of two ellipses, rotated \SI{90}{\degree} with respect to each other, as shown in \cref{fig:biaxial_island:geometryTypical}. This gives the geometry two degrees of freedom, namely the long and short axis of the ellipse. We will define the roundness $\rho$ as the ratio between the long and short axis of the ellipses, and the `overall size' $L$ is simply equal to the long axis.
\begin{figure}
    \centering
    \includegraphics[width=0.3\columnwidth]{Figures/biaxial_island/Geometry/geomPlus55.png}
    \caption{Typical geometry of the biaxial island under investigation, in this specific case for 55x\SI{100}{\nano\metre} ellipses, i.e. $(\rho, L)=(0.55, \SI{100}{\nano\metre})$.}
    \label{fig:biaxial_island:geometryTypical}
\end{figure}

\subsubsection{Energy barrier}
To determine the energy barrier $E_{barrier}$, the energy landscape as a function of magnetization angle must be calculated. If one were to simply set the magnetization $\vb{M}$ of the entire island in one direction and call it a day, the calculated energy landscape would be flat, due to the absence of the demagnetizing field caused by the shape anisotropy of the magnetic body.~\cite{Nonmonotonic_reversal} Thus the magnetization should first be relaxed, which can be done using the \mumax{} command \code{minimize()}. However, this would result in a complete relaxation to an energy minimum, hence not yielding any useful information on the energy barrier. To circumvent this issue, an external magnetic field is applied to keep the average magnetization $\frac{1}{A}\int_A \vb{M}\,dA$ close to the initial direction. \par
The procedure is therefore as follows. First the magnetization of the entire island $\vb{M}$ is initialized under one specific angle $\theta$. Then an external magnetic field $B_{ext}$ is applied along that same direction, and the magnetization is relaxed using \code{minimize()}. The internal energy, responsible for the energy landscape, is then equal to $E_{total} - E_{Zeeman}$. This procedure is repeated for different angles $\theta$ to generate an energy profile from which the energy barrier can be deduced. The grid size for these numerical calculations is \SI{1}{\nano\metre} for maximum accuracy.

\paragraph{Optimal simulated external field magnitude}
The procedure as described above makes use of an external magnetic field. It is important to know what the magnitude of this external field should optimally be to determine the energy barrier. To this end, a simulation is carried out in which both the angle $\theta$ and the external magnetic field $B_{ext}$ are varied. In \cref{fig:barrierLandscape-sweepBext} the relaxed magnetization angle is plotted for different magnitudes of the external magnetic field $\abs{\vb{B_{ext}}}$, with the angle of this magnetic field taken at 64 equally spaced values from \SIrange{0}{90}{\degree}.
\begin{figure}
    \centering
    \includegraphics[width=0.9\textwidth]{Figures/biaxial_island/BarrierLandscape/Plus_65_B25-0.001-div4_a128Pi_plotOptimized.pdf}
    \caption{Energy landscape between \SIrange{0}{90}{\degree} for various external magnetic field magnitudes $\abs{\vb{B_{ext}}}$. The external field angle was varied uniformly from \SIrange{0}{90}{\degree} in 64 steps.}
    \label{fig:barrierLandscape-sweepBext}
\end{figure}

Interesting findings:

- Firstly, at very high magnetic fields, the energy landscape is nearly flat as explained earlier.~\cite{Nonmonotonic_reversal}

- Secondly, at an angle of exactly \SI{45}{\degree}, the magnetization remains at the maximum of the energy landscape, which is an unstable equilibrium.

- Thirdly, for lower and lower magnetic field values, the situations for angles very close to \SI{45}{\degree}, the magnetization relaxes to an angle closer to that of the energy minimum. This is to be expected, because lower magnetic fields will have more difficulty keeping the magnetization pointed in the same direction, against the anisotropy.

- Fourthly or finally idk, for lower magnetic fields the energy barrier lowers until it reaches a limit value. This is because high external magnetic fields prevent the relaxation of the edges of the geometry, instead keeping them pointed in roughly the same direction. It is the relaxation of the magnetization angle in a non-uniform way which causes the anisotropy, so forcing the magnetization in a certain direction (like with a high external field) will prevent this from happening, thus increasing the energy and lowering the fictional variant of the energy barrier.

\paragraph{Energy barrier as function of roundness}
See \cref{fig:barrier-cell_size-1nm} blue curve for \SI{128}{\nano\metre}.

\paragraph{Energy barrier as function of overall size}
See \cref{fig:barrier-cell_size-1nm} all different curves

\paragraph{Numerical error: influence of cell size on energy barrier}
When performing longer simulations, it is advantageous to use as large a cell size as possible, to minimize the total amount of cells in the simulation. Increasing the size of the cells does however also increase the numerical error originating from the discretization of the grid. The size of this numerical error was examined by determining the energy barrier for different shapes (varying roundness and overall size) and for different cell sizes of \SI{4}{\nano\metre}, \SI{2}{\nano\metre} and \SI{1}{\nano\metre}. The results of this calculation are shown in \cref{fig:barrier-cell_size}.
\begin{figure}
     \centering
     \begin{subfigure}[b]{0.75\textwidth}
         \centering
         \includegraphics[width=\textwidth]{Figures/biaxial_island/Barrier/Plus_16-128_0.1-1_aPi4_B0.001_cell1nm.pdf}
         \caption{\SI{1}{\nano\metre}}
         \label{fig:barrier-cell_size-1nm}
     \end{subfigure}
     \hfill
     \begin{subfigure}[b]{0.75\textwidth}
         \centering
         \includegraphics[width=\textwidth]{Figures/biaxial_island/Barrier/Plus_16-128_0.1-1_aPi4_B0.001_cell2nm.pdf}
         \caption{\SI{2}{\nano\metre}}
         \label{fig:barrier-cell_size-2nm}
     \end{subfigure}
     \hfill
     \begin{subfigure}[b]{0.75\textwidth}
         \centering
         \includegraphics[width=\textwidth]{Figures/biaxial_island/Barrier/Plus_16-128_0.1-1_aPi4_B0.001_cell4nm.pdf}
         \caption{\SI{4}{\nano\metre}}
         \label{fig:barrier-cell_size-4nm}
     \end{subfigure}
        \caption{Energy barrier for different cell sizes. The legend represents the length of the long axis of the ellipses.}
        \label{fig:barrier-cell_size}
\end{figure}
\begin{figure}
    \centering
    \includegraphics[width=0.9\columnwidth]{Figures/biaxial_island/Barrier/Plus_100_0.1-1_aPi4_B0.001.pdf}
    \caption{Energy barrier for different cell sizes, for a long ellipse axis of \SI{100}{\nano\metre}.}
    \label{fig:barrier-cell_size-100nm}
\end{figure}
The figure obtained with cells of \SI{1}{\nano\metre} is quite smooth, so it is justified that this size was used in the previous paragraphs. For \SI{2}{\nano\metre} cells, the curve becomes rougher, but the overall shape remains very similar to that of \SI{1}{\nano\metre} cells. Cells of \SI{4}{\nano\metre}, however, yield a very rough and almost stair-like energy landscape, which clearly indicates that this size is too large. \par
It may also be noted that for the \SI{4}{\nano\metre} cells, there are no values for the lowest roundnesses for long ellipse axes of \SI{16}{\nano\metre} or \SI{32}{\nano\metre}, because for these situations the short axis is smaller than the grid size, which yields an empty simulation geometry and thus no results. \par
Everything taken into account, it seems that the optimal compromise between simulation speed and accuracy, is a \SI{2}{\nano\metre} cell size, which will be used to perform longer simulations. For the 65x100nm geometry, a \SI{4}{\nano\metre} cell size is probably still acceptable since the energy barrier is similar, and since this will result in a 4x speed increase this will be used. For the perfectly round situation, the difference is very large so it is instructive to use \SI{2}{\nano\metre} or even \SI{1}{\nano\metre}.





% FROM HERE ON THINGS ARE NOT VERY STRUCTURED
\subsection{Random thermal switching}
\subsubsection{Preparation}
\textbf{Step 1: find energy barrier} \\
The procedure used is to set both $m$ and $B_{ext}$ at a certain angle and then \code{minimize()}. Both the magnetic field $B_{ext}$ and the angle $\theta$ are then varied, with $B_{ext}$ going from very high to very low values, and $\theta=0\dots\pi/2$ in steps of $\pi/4$. One could increase the amount of $\theta$-steps, but it is observed that an angle of $\pi/4$ remains magnetized at that angle regardless of the external field, at the very least until $B_{ext} = \SI{0.00153}{\tesla}$. It is however an unstable equilibrium and any slight deviation from this angle will quickly relax to either \SI{0}{\degree} or \SI{90}{\degree} at low fields.
One could even omit the part where theta goes from $\pi/4$ to $\pi/2$ due to symmetry reasons.
All of $\theta = k\frac{\pi}{4} , k\in\mathbb{Z}$ are stable.
At high fields everything is stable and just uniform, which makes the energy landscape flat.

\textbf{Step 2: random switching over \SI{100}{\nano\second}} \\
In GYP-18~\cite{GYP-18} the formula
\begin{equation}
    t_i = -\frac{1}{f_0} \exp(\frac{\Delta E_i}{k_B T}) \ln(1-P_i)
    \label{eq:Switching_time}
\end{equation}
appears, with $f_0 = \SI{e12}{\hertz}$ attempt frequency, and $P_i$ random between 0 and 1. The average value of $-\ln(1-P_i)$ is then $\ln(4) \approx 1.4$.

To get an average $t_i = \SI{100}{\nano\second}$ at $T=\SI{300}{\kelvin}$, one should therefore have
\begin{equation}
    \Delta E_i = k_B T \ln(\frac{f_0 t_i}{\ln(4)}) = 11.19 k_B T = \SI{0.310}{\electronvolt} \mathrm{.}
\end{equation}
The barrier height of a \SI{65}{\nano\metre} elliptic plus-sign geometry is \SI{0.346}{\electronvolt}.
Idea: perhaps attempt frequency is the sinusoidal pattern in the long simulations, with period of about \SI{0.5}{\nano\second}. Observation: This worked with the previous method where $B_{ext}$ was not varied, but with the new (correct) method the $f_0 = \SI{e12}{\hertz}$ works very well again.
Also there is another formula in mumax3~\cite{MuMax3} for uniaxial anisotropy.

Observation: with a barrier height approximately equal to $\SI{0.346}{\electronvolt}$, a switching speed of about \SI{100}{\nano\second} is observed.

Question: should $\alpha$ be 0.1 always, i guess this is just a case of ``0.1 works so we use 0.1''
Answer: it's better to use 0.01

\subsubsection{Results}
Step 1: find energy barrier \\
With perfectly circular geometry (i.e. \\
\code{geometry := Ellipse(100e-9, 100e-9)} \\
\code{geometry = geometry.Add(Ellipse(100e-9, 100e-9).RotZ(Pi/2))}) \\
there is still shape anisotropy, but with opposite sign (what are maxima in actual plus-figures are now minima)

Findings (for \SI{100}{\nano\metre} long axis ellipse plus-sign):
at \SI{45}{\nano\metre} the preferred directions switch from the axes (\SI{0}{\degree}) to in between the axes (\SI{45}{\degree}),
at \SI{65}{\nano\metre} the anisotropy with preferred \SI{0}{\degree} direction is the strongest

After running for \SI{1}{\micro\second} at \SI{300}{\kelvin}, with shape \SI{65}{\nano\metre} by \SI{100}{\nano\metre} ellipses in plus shape, the following angles were observed (\cref{fig:switching-alpha}):


\subsubsection{Counting switches}
Since the counting of switches is ambiguous, several different ways of counting them were performed:

1) Counting every crossing of a \SI{90}{\degree} line, e.g. a \SI{270}{\degree} switch is counted as 3 individual switches.

2) Counting every visible switch, with switches larger than \SI{90}{\degree} counted as 1 switch.

3) Looking every \SI{5}{\nano\second} and seeing if the magnetization direction changed without looking at the detailed magnetization in between those timestamps. This excludes 'spikes'.

Formula for $f_0$ derived from \eqref{eq:Switching_time}:
\begin{equation*}
    f_0 = \frac{N \ln(4)}{\Delta t} \exp(\frac{E_{barrier}}{k_B T})
\end{equation*}
with $\Delta t$ the simulation time and $N$ the number of switches.

% TODO: This table is too wide, but currently serves as a sort of data storage
\begin{adjustbox}{center}
% \begin{table}[b!]
    \centering
    \begin{tabular}{c|c|c|c|c|c|c|c|c|c}
        Geom [nm] & Barrier [meV] & T [K] & $\Delta t$ [ns] & (1) & (2) & (3) & f0 (1) & f0 (2) & f0 (3) \\
        \hline
        49x100 & 2nm: 24.7 & 300 & 100 & 57 & 37 & 6 & \SI{2.06e9}{} & \SI{1.33e9}{} & \SI{2.16e8}{} \\
        \hline
        65x100 & 4nm: 155.0 & 273 & 1000 & 13 & 13 & 9 & \SI{1.31e10}{} & \SI{1.31e10}{} & \SI{9.06e9}{} \\
        65x100 & 4nm: 155.0 & 300 & 1000 & 15 & 11 & 7 & \SI{8.35e9}{} & \SI{6.12e9}{} & \SI{3.89e9}{} \\
        65x100 & 4nm: 155.0 & 350 & 1000 & 51 & 41 & 30 & \SI{1.21e10}{} & \SI{9.69e9}{} & \SI{7.09e9}{} \\
        65x100 & 2nm: 286.0 & 350 & 500 & 0 & 0 & 0 & 0 & 0 & 0 \\
        65x100 & 3.125nm: TBD & 350 & 1000 & TBD & TBD & TBD & TBD & TBD & TBD \\
        \hline
        100x100 & 4nm: 241.1 & 300 & 1000 & 0 & 0 & 0 & 0 & 0 & 0 \\
        100x100 & 4nm: 241.1 & 350 & 1000 & 5 & 5 & 5 & \SI{2.06e10}{} & \SI{2.06e10}{} & \SI{2.06e10}{} 
    \end{tabular}
    % \caption{For $\alpha=0.01$.}
    \label{tab:switching-summary}
% \end{table}
\end{adjustbox}

\begin{figure}
     \centering
     \begin{subfigure}[b]{0.75\textwidth}
         \centering
         \includegraphics[width=\textwidth]{Figures/biaxial_island/Switching/65x100_300K_alpha0.1_1µs_4nm.pdf}
         \caption{$\alpha=0.1$}
         \label{fig:switching-alpha-0.1}
     \end{subfigure}
     \hfill
     \begin{subfigure}[b]{0.75\textwidth}
         \centering
         \includegraphics[width=\textwidth]{Figures/biaxial_island/Switching/65x100_300K_alpha0.01_1µs_4nm.pdf}
         \caption{$\alpha = 0.01$}
         \label{fig:switching-alpha-0.01}
     \end{subfigure}
        \caption{Switching for different $\alpha$, with grid discritization of \SI{4}{\nano\metre} at \SI{300}{\kelvin}, for 65x100.}
        \label{fig:switching-alpha}
\end{figure}

\begin{figure}
     \centering
     \begin{subfigure}[b]{0.49\textwidth}
         \centering
         \includegraphics[width=\textwidth]{Figures/biaxial_island/Switching/65x100_300K_alpha0.01_1µs_4nm.pdf}
         \caption{\SI{300}{\kelvin}, 65x100}
         \label{fig:switching-temp-300-65x100}
     \end{subfigure}
     \hfill
     \begin{subfigure}[b]{0.49\textwidth}
         \centering
         \includegraphics[width=\textwidth]{Figures/biaxial_island/Switching/65x100_350K_alpha0.01_1µs_4nm.pdf}
         \caption{\SI{350}{\kelvin}, 65x100}
         \label{fig:switching-temp-350-65x100}
     \end{subfigure}
     \begin{subfigure}[b]{0.49\textwidth}
         \centering
         \includegraphics[width=\textwidth]{Figures/biaxial_island/Switching/100x100_300K_alpha0.01_1µs_4nm.pdf}
         \caption{\SI{300}{\kelvin}, 100x100}
         \label{fig:switching-temp-300-100x100}
     \end{subfigure}
     \hfill
     \begin{subfigure}[b]{0.49\textwidth}
         \centering
         \includegraphics[width=\textwidth]{Figures/biaxial_island/Switching/100x100_350K_alpha0.01_1µs_4nm.pdf}
         \caption{\SI{350}{\kelvin}, 100x100}
         \label{fig:switching-temp-350-100x100}
     \end{subfigure}
        \caption{Switching for different temperatures, with grid discritization of \SI{4}{\nano\metre} and $\alpha = 0.01$.}
        \label{fig:switching-temp}
\end{figure}
\begin{figure}
    \centering
    \includegraphics[width=0.9\columnwidth]{Figures/biaxial_island/Switching/49x100_300K_alpha0.01_100ns_2nm.pdf}
    \caption{Switching for a long ellipse axis of \SI{49}{\nano\metre}, which has a very low energy barrier, with grid discretization of \SI{2}{\nano\metre} at \SI{300}{\kelvin} and $\alpha = 0.01$.}
    \label{fig:switching-49x100-300}
\end{figure}


\newpage
\bibliographystyle{IEEEtran}
\bibliography{bibliography/bibliography.bib}


\end{document}
